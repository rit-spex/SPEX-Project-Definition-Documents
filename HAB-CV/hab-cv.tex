%% *************************************************************************
%%
%% This is a derivative work of the RIT Space Exploration Standard defining
%% guidelines for content and formatting of project design documents.
%%
%% This document uses IEEEtran.cls, the official IEEE LaTeX class
%% for authors of the Institute of Electrical and Electronics Engineers
%% (IEEE) Transactions journals and conferences.
%%
%% *************************************************************************

\documentclass[conference]{IEEEtran} % http://www.ctan.org/pkg/ieeetran
\usepackage{blindtext} % enable placeholder text generator
\usepackage{graphicx} % enable toolbox for embedding figures and pictures
\usepackage{nomencl} % enable package for adding a list of variables and constants at the beginning, aka "nomenclature"
\usepackage{siunitx} % enable package for easily formatting units
\usepackage{hyperref} % enable package for cross-referencing figures, sections, references etc.
% how to use hyperref: http://www2.washjeff.edu/users/rhigginbottom/latex/resources/lecture09.pdf
\usepackage[T1]{fontenc} % change text encoding to make it more crisp
\usepackage{etoolbox} % enable conditionals for help text
\usepackage{booktabs} % make beautiful tables!

\title{On-Board Image Processing and Computer Vision Techniques for Vegetation Density Mapping and Other Experiments on Low-Cost Consumer Electronics}

\author{
   \IEEEauthorblockN{% This block is for author Names.
    Jeff~Maggio\IEEEauthorrefmark{1},
    Philip~Linden\IEEEauthorrefmark{2},
    T.J.~Tarazevits\IEEEauthorrefmark{3}
  }
  \IEEEauthorblockA{% This block is for the author Affiliations, aka department and university
    RIT Space Exploration, Rochester Institute of Technology \\ %\\ starts a new line
    Rochester, N.Y. \\
    Email:
    \IEEEauthorrefmark{1}jxm9264.rit.edu,
    \IEEEauthorrefmark{2}pjl7651@rit.edu,
    \IEEEauthorrefmark{3}tjt3085@rit.edu
  }
}
% page header for pages other than cover page
\markboth{Vegetation Density Mapping with Computer Vision}%
{Maggio \MakeLowercase{\textit{et al.}}: RIT Space Exploration}

\begin{document}
\maketitle%
% correct bad hyphenation here, separated by spaces
\hyphenation{explor-ation}

\begin{abstract}
    Advanced on-board image processing is a foundational component of a wide range of future space science and Earth observation missions.
    Extending these techniques to include computer vision opens the door to even more opportunities for science.
    It is critical to develop these techniques on low-cost, consumer hardware platforms so that the missions need not require expensive, specialized systems for every experiment.
    Demonstrating these systems are themselves opportunities for science as well.
\end{abstract}

\section{Introduction}
\label{sec:introduction}
  % The introduction is a place to give background and context before diving into the subject matter.
  % Establish context for the work you are about to propose and the main ideas of the proposition itself.

\IEEEPARstart{I}{mage} processing has long been a critical element of Earth observation and space science.
In recent years, the capabilities of inexpensive consumer electronics and computers have reached a point where advanced image processing can be performed on-board with lightweight, low-power computers.
This has opened the door for low-cost, rapid development experiment payloads and platforms such as high altitude balloons, drones, and small satellites.
Usually these platforms have limited communications bandwidth, so on-board processing may be used to significantly reduce the amount of data transferred back to ground without losing the information that the images contain.

Computer vision (CV) is defined as the automatic extraction, analysis and understanding of useful information from a single image or a sequence of images.
CV is the realm between image processing and computer science where useful information is identified, extracted, and interpreted from images without human input.
In addition to edge detection and other transforms applied to the pixel arrays directly, deeper and more abstract algorithms to interpret the contents of the images continue to mature in the field of machine learning.
These algorithms are trained, or iteratively tuned with a large set of data, to classify objects or cluster multivariate data from an arbitrary set of inputs.

Naturally, any implementation of image processing or CV for space science must be tested in a flight setting.
As this technilogy is developed, all tests are themselves opportunities for science.
This Project Definition Document considers the logistics of this development in addition to discussing a number of experiments that may be conducted as tests or end-user applications of on-board image processing and computer vision with low-cost, consumer electronics.

\section{Primary Objective}
\label{sec:primary-obj}
  % At the end of the day, whether the project ``succeeds'' or ``fails'' is judged against the objectives it sought to meet.
  % Note that results that contradict expectations/hypotheses are not failures if the scientific \& engineering methods are followed along the way.
  % Sometimes our expectations are wrong and that can be just as successful as getting data we thought we'd see.
  % What matters are what questions you intend to answer.
  % This is the main purpose or main goal the project hopes to achieve.

The ideal result of developing robust image processing and computer vision techniques on flight electronics is a payload module for a high altitude balloon, small satellite or other flight system which is capable of reducing a video or image stream into a stream of processed useful information which can be relayed back to a ground station or efficiently saved to system memory.

While it is obvious that more powerful (and more expensive) electronics are capable of more advanced processing, the goal of this project is to push the limits of what entry-level hardware is capable of.
This way, software development takes the lead over hardware development.
Since software can be reused between flights, loss of mission is not critical with low-cost flight electronics.

\section{Secondary Objectives}
\label{sec:secondary-obj}
Second to generalized platform development for an imaging and computer vision payload module is the science that payload would actually conduct.
Every flight is a new opportunity to collect data, perform an experiment, or demonstrate new technology.
Every module test will have a science goal in addition to any technology advancement goals.
Specific experiment ideas are discussed in \autoref{sec:payloads}.

% \begin{table*}
% % this table is too wide for the two-column format, so we let it expand across both columns
% % we haven't told LaTeX where to put this so it'll find the best place.
%     \caption{Relative detail expected at each stage of project development.}
%     \centering
%     \begin{tabular}{@{}llcc@{}}
%         % READ THIS!! https://www.inf.ethz.ch/personal/markusp/teaching/guides/guide-tables.pdf
%         \toprule % line on top external edge of table
%         % Separate cells in a row with &, move to the next row with \\
%         Document & Purpose & Contributors & Destination \\
%         \midrule % line separating two internal rows
%         Project Definition Document & To define the goals and requirements of a SPEX project. & 2--3 people & SPEX Archive \\
%         Project Plans & Specific plans for when work is to be done (Gantt charts) & 2--3 people & Project Repository \\
%         Design Reviews & To review designs before work is started. & 6--8 people & Project Repository \\
%         Test Procedures & Specific instructions and data logs for tests. & 3--4 people & Project Repository \\
%         User Manual & Instructions for future users of project deliverabels. & 3--4 people & Project Repository \\
%         Posters \& Presentations & Materials for sharing projects with the public. & 5--6 people & Project Repository \\
%         Technical Report & Final technical summary of work done and results. & 6 or more & SPEX Archive, Conferences \& Journals \\
%         % LaTeX doesn't really like multi-line cell contents. Try to keep the text in each cell concise!
%         \bottomrule
%     \end{tabular}
% \label{tab:long-example}
% \end{table*}

\section{Benefit to SPEX}
\label{sec:benefit}
% One of the core values of SPEX is to provide opportunities for academic and professional growth for its members,
% and to challenge them with interesting projects.
% In this section, explain how the project would benefit SPEX members as students,
% space enthusiasts, and young professionals.




% Below I have used subsections to identify key ideas in this section. These particular subsections are not required as part of the SPEX Standard, but serve as an example of using subsections in a text.

\subsection{Mindset}
\label{subsec:mindset}


\subsection{Traceability}
\label{subsec:traceability}


\subsection{Accessibility}
\label{subsec:plug-n-play}
  % Note below that LaTeX uses weird formatting when it comes to quotation marks.
  % The style below is correct to display forward quotes `` at the start of the phrase and backquotes '' at the end.



\section{Implementation}
\label{sec:implementation}
  % What path do you anticipate the project to take?



\subsection{Deliverables}
\label{subsec:deliverables}
  % When all is said and done, what will you have to show for it?
  % Examples: Hardware, software, poster, ImagineRIT demo, presentations, technical papers...


\subsection{Milestones}
\label{subsec:milestones}
  % Be as detailed as you can, but it's okay if there are unknowns.
  % At the very least, specify how many semester you expect the project to take until it reaches completion.


% \begin{table}[hb!]
%     % the "h" in these brackets tells LaTeX to put the table Here. Try [t] for top and [b] for bottom,
%     % or [hbp] for "here, or if you can't do that put it at the bottom of the page, or if you can't do that put it on its own page.
%     % Here we've also used an "!" to yell at LaTeX to DO THIS OR ELSE!
%     \caption{Notional timeline of Project Milestones.}
%     \centering
%     \begin{tabular}{@{}cll@{}}
%     % the letters here ^^^^ designate the columns.
%     % (l=left align, c=center, r=right align)
%     % the weird @{} thingies tell LaTeX to not have left-right padding between cells
%     % so cells butt up right against the edge
%     \toprule
%     Phase & Task & Duration \\
%     \midrule
%     1 & Review existing designs and materials & 2 weeks or less\\
%     2 & Subsystem development & 6 weeks \\
%       & Order PCB design and/or assembly & 6 weeks \\
%       & Review changes and order materials & 2 weeks or less\\
%       & Testing of individual subsystems & 2 weeks \\
%     3 & System assembly & 1 week  \\
%     4 & System testing & 2 weeks  \\
%     5 & Generate documentation and delivery to SPEX & 1 week  \\
%     \bottomrule
%     \end{tabular}
% \label{tab:short-example}
% \end{table}

\section{Externalities}
  % Things not directly related to the work or outcomes, but related to the project as a whole.
\subsection{Prerequisite Skills}
  % Which skills do team members need to have before work can start (not including skills that will be learned ``on the job'')?


\subsection{Funding Requirements}
  % Estimate costs that would be needed to meet objectives.


\subsection{Faculty Support}
  % Identify faculty that will be involved (or would need to be involved) to meet objectives.
  % Note that if a professor is the Principal Investigator (P.I.) for a project, there still needs to be a student as the SPEX Project Champion.


\subsection{Long-Term Vision}
\label{sec:vision}


\section{Applications and Experiments}
\label{sec:payloads}
\subsection{WUAP: where u at plants?}
Vegetation density and NDVI. Mapping density with gps data.

\subsection{WTFbiome: Wayfinding TransFormations and Biome identification}
scene (biome) identification.
morphological transformations to apply images to spatial coordinates.

\subsection{SUP: Stereo groUnd mapping and Photometry}
stereo 3d ground mapping and photometry.

\subsection{PIP: Passive Instrument Payload}
as an instrument: horizon detection. star tracking.

\section*{Acknowledgements}
The authors would like to thank all contributors to the Python open source community, especially the authors of the OpenCV and Numpy libraries.

\bibliographystyle{IEEEtran}
\bibliography{sample-with-examples}

\onecolumn
\appendices{}


\end{document}
