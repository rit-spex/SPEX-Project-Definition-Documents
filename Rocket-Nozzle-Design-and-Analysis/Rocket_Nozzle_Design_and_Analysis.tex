%% *************************************************************************
%%
%% This is an RIT Space Exploration Standard defining guidelines for content
%% and formatting of project design documents.
%%
%% This document uses IEEEtran.cls, the official IEEE LaTeX class
%% for authors of the Institute of Electrical and Electronics Engineers
%% (IEEE) Transactions journals and conferences.
%%
%% *************************************************************************

%% *************************************************************************
% LaTeX REFERENCES
% ----------------
%   Intro to LaTeX: http://www.rpi.edu/dept/arc/docs/latex/latex-intro.pdf
%   Comprehensive LaTeX symbol list: http://tug.ctan.org/info/symbols/comprehensive/symbols-a4.pdf
%% *************************************************************************

% tell \LaTeX what kind of formatting to use
\documentclass[conference]{IEEEtran} % http://www.ctan.org/pkg/ieeetran
% enable placeholder text generator
\usepackage{blindtext}
% enable toolbox for embedding figures and pictures
\usepackage{graphicx}
% enable package for adding a list of variables and constants at the beginning, aka "nomenclature"
\usepackage{nomencl}
% enable package for easily formatting units
\usepackage{siunitx}
% enable package for cross-referencing figures, sections, references etc.
% how to use hyperref: http://www2.washjeff.edu/users/rhigginbottom/latex/resources/lecture09.pdf
\usepackage{hyperref}
% change text encoding to make it more crisp
\usepackage[T1]{fontenc}
% enable conditionals for help text
\usepackage{etoolbox}

\usepackage{float}

\usepackage{array}

\usepackage{booktabs}

\usepackage{tabularx}

% initialize nomenclature package
\makenomenclature{}

% set title. choose something as descriptive and precise as possible. Descriptive > sounding cool. remember this!
\title{Rocket Nozzle Design and Analysis}


\author{
  % List the authors of the design document. The Champion should go first.
  % The \$~\$ markers tell \LaTeX{} to treat the text inside to be treated as a math expression. This way you can use operators like \textcaret{} to place characters as superscripts.
  % Some \LaTeX{} templates handle the author block in different ways. For example, the \href{http://www.worldscientific.com/worldscinet/jai}{Journal of Astronomical Instrumentation} requires the authors' addresses and emails to be included as well.
  % The \textbackslash{}thanks command puts the contents inside those brackets in a footnote at the bottom of the first page. Technically speaking, \textbackslash{}thanks is just a specially formatted footnote.
  % IEEE also has a ``long form'' author block for many authors. Check here for more information:
  % \url{https://tex.stackexchange.com/questions/156523/multiple-authors-with-common-affiliations-in-ieeetran-conference-template}
  % Read here for a more advanced options to modifying footnotes in the author block:  \url{http://tex.stackexchange.com/questions/826/symbols-instead-of-numbers-as-footnote-markers}
  %   Here, we use the IEEE long-form author block.
  \IEEEauthorblockN{% This block is for author Names.
    James~Emerson~Parkus\IEEEauthorrefmark{1},  %the number in the bracket is a reference number to identify this footnote. \LaTeX will figure out what symbol to put there.
    David~Breen\IEEEauthorrefmark{2}
  }
  \IEEEauthorblockA{% This block is for the author Affiliations, aka department and university
    RIT Space Exploration, Rochester Institute of Technology \\ %\\ starts a new line
    Rochester, N.Y. \\
    Email:
    \IEEEauthorrefmark{1}jep7631@rit.edu,
    \IEEEauthorrefmark{2}djb1410@rit.edu
  }
  %%   Below, we use the short-form author block and basically hack it to suit our needs.
  % Philip~Linden$^{*\dagger}$%
  %   \thanks{$^{*}$Project Champion}%
  %   \thanks{$^{\dagger}$BS/MEng '17, Mechanical Engineering},
  % Austin~Bodzas$^{\ddagger}$%
  %   \thanks{$^{\ddagger}$BS '19, Computer Science},
  % Drew~Walters$^{\S}$%
  %   \thanks{$^{\S}$BS '18, Mechanical Engineering Technology},
  % T.J.~Tarazevits$^{**}$%
  %   \thanks{$^{**}$BS '19, Game Design \& Development}%

  %%   If there are many authors, consider using symbolic, numeric (aka arabic),  alphabet footnotes or a combination thereof.
  %% the recommended order for symbolic footnotes is
  %%   (1) asterisk        *   *
  %%   (2) dagger          †   \dagger
  %%   (3) double dagger   ‡   \ddagger
  %%   (4) section symbol  §   \S
  %%   et cetera. For higher counts, use 2x symbols (1)-(4) (i.e. (5) two asterisks **). Keep cycling through (1)-(4) using 3x, 4x, and so on.
  %%   Note that these symbol codes work in math mode and text mode.
  %%   There are ways to make LaTeX do this for you, but it is more advanced and not entirely necessary, especially for short author lists. Not worth the hassle, in my opinion.
}
% page header for pages other than cover page
\markboth{Project Design Document Standard}%
{Parkus \MakeLowercase{\textit{et al.}}: RIT Space Exploration}

% Initial setup is over, start building the document itself
\begin{document}
\maketitle%
% correct bad hyphenation here, separated by spaces
\hyphenation{explor-ation}

\begin{abstract}
  This project focuses on the design of a large-scale rocket nozzle capable of producing 100 Newtons of thrust. There will be a test stand used to measure the thrust force by method of a load cell.
  There will be sensors placed in strategic locations that will assist in the data analysis to allow the team to understand conditions of the fluid as it passes through the nozzle and exits. The results
  of the experiment will be summarized in a final report and presented at Imagine RIT, depending on the completion date.
\end{abstract}

\label{sec:nomenclature}
\newcommand{\nomunit}[1]{%
\renewcommand{\nomentryend}{\hspace*{\fill}#1}}
\renewcommand{\nompreamble}{
    % If you include mathematical expressions or express variables in the design document, list them with their corresponding definitions here as a list.
    % The two lines below make it look nice when defining units/values to constants.

    % Note that math terms and non-math terms are separated and alphabetized, regardless of the order in which they are defined. (Recall terms \$like this\$ are in the math environment)
    % Read more about advanced nomenclature formatting here:\\
    % \url{https://www.sharelatex.com/learn/Nomenclatures}
  }
\nomenclature{RIT}{Rochester Institute of Technology}
\nomenclature{SPEX}{RIT Space Exploration}
\nomenclature{PDD}{Project Design Document}
\nomencalture{BoM}{Bill of Materials}
% Below are examples of using nomenclature for math symbols and constants or units
% \nomenclature{$\dot{m}$}{Mass flow rate
%   \nomunit{\,\si{\kilo\gram\per\second}}}
% \nomenclature{$c$}{Speed of light
%  \nomunit{\,\SI{2.9979e8}{\meter\per\second}}}
\printnomenclature{}

\section{Introduction}\label{sec:introduction}
\IEEEPARstart{T}{he} project will be concerned with the design, fabrication, and experimentation of a cold gas propulsion system. This is a propulsion system
that is solely based off momentum transfer with little to no heat considerations. The project should have a team consisting of 10 to 15 members willing to dedicate
5 to 10 hours a week to this project. The end result of the project will be a propulsion system, mounted on a test stand ready to test when the appropriate propellant
is attached.

\section{Primary Objective}\label{sec:primary-obj}
The primary goal of this project is to use the analytical rocket equations to design a conical rocket nozzle to produce 100 Newtons of thrust.
The team will design a new stand that is capable of securely fastening a propulsion test of 200 Newtons.
Thus providing a factor of safety of 2.0. The tests will be performed with Air or Nitrogen so the extensive tabulated data can be used in the performance analysis of the nozzle.

\section{Benefit to SPEX}\label{sec:benefit}
The benefit to SPEX would be the further education in the design of rocket nozzles using the analytical rocket equations. The experiment platform would
be a great exhibit for visiting companies or Imagine RIT where SPEX would be on display. This project would be
the first time that the propulsion team has fully designed and operated a rocket nozzle. Using these equations correctly is very complicated and important for
propulsion engineers to understand. The project would have a full life cycle which includes design, manufacture, test, and reporting the results.

\section{Implementation}\label{sec:implementation}
The project will begin by informing new members on past projects. The team will discuss the events that went well, and the reasons some did not. The objective is to prevent
the same problem occuring twice. Each team member will be instructed to read chapter 3 out of the \textit{Rocket Propulsion Elements} book written by Sutton.
Then the team will start with the nozzle design. The nozzle design process begins by defining the requirements, then using the analytical rocket equations, find the
physical dimensions of the nozzle so it can be 3D modeled and sent to a manufacturer. During the nozzle design process, a second part of the team will design the test stand that can adequately contain the
propulsion system. Then the team will manufacture the nozzle and test stand. After testing, if time permits, the team will use a schlieren lensing setup to
get a very good look at the fluid flowing out of the nozzle.

\subsection{Deliverables}\label{subsec:deliverables}
The physical deliverables for this project are the test stand and rocket nozzle. The test stand and rocket nozzle can be shown to faculty and other
bodies that could support SPEX in some fashion. The test stand and nozzle will be stored so it can be demonstrated when necessary, such as an introduction to SPEX
engineering to students, faculty, or others.

The non-physical deliverables would be a poster that exhibits the work completed and the results of the experiment. There will be a final report that outlines
the method used for designing the nozzle and manufacturing the nozzle and test stand. The report will be written in the IEEE standard and can be submitted to various
organizations for funding other outreach opportunities.

\subsection{Milestones}\label{subsec:milestones}
The milestones for this project are, in chronological order; nozzle design, test stand design, nozzle fabrication, test stand fabrication, testing, and final report.
The nozzle design and fabrication proceed the test stand because the test stand must be designed to fit the nozzle, the nozzle should not be designed to fit the test stand.
The following, is a proposed project schedule.

\begin{table}\label{tab:proposed-timeline}
  \centering
  \begin{tabularx}{\columnwidth}{@{}cXc@{}}
    \multicolumn{2}{ c }{\textbf{Proposed Timeline}} \\ \toprule
    Week & Description \\ \midrule
    1 & Team Introductions and project planning \\
    2 & Nozzle Design \& Test Stand Design\\
    3 & Nozzle Design \& Test Stand Design\\
    4 & Nozzle Design \& Test Stand Design\\
    5 & Nozzle Design \& Test Stand Design\\
    6 & Order nozzle from manufacturer \& Test Stand BoM Purchase \\
    7 & Test Stand Fabrication \\
    8 & Test Stand Fabrication \\
    9 & Order Gas \\
    10 & Setup experiment \\
    11 & Test \#1 \\
    12 & Test \#2 \& Schlieren Lensing \\
    13 & Data Analysis \\
    14 & Final Report \\
    \bottomrule
  \end{tabularx}
\end{table}

\begin{}
\section{Externalities}
  % Things not directly related to the work or outcomes, but related to the project as a whole.
\subsection{Prerequisite Skills}
No skills are required. All necessary skills will be taught. These skills include; Differential \& Integral Calculus (to a degree so understanding the rocket equations is possible), MATLAB coding, engineering decisions, machining design choices, and using the rocket equations. The rocket equations
require a basic knowledge of calculus to understand. Rates of change and integration show up very regularly in this type of physics and mathematics. MATLAB
is a very usefull resource. MATLAB will be used for the data collection and the data analysis. It's ability to handle vectors and large datasets is unparalleled.
A very basic understanding of MATLAB would be great, but the team plans to host MATLAB training sessions so everyone can be brought to a rudimentary understanding
level of MATLAB\@. The machining for this project will mainly concern the test stand, the nozzle will most likely be outsourced due to high machining complexity.

\subsection{Funding Requirements}
The project will most likely require a budget of \$400. There will be material costs for the test stand and the nozzle.
Then there will be the cost of a proper load cell. There will also be the cost of the necessary sensors to measure
 the proper characteristics of the fluid flowing through the nozzle.

Measuring the thrust is extremely important to this project. The accuracy of this measurement must be high enough that it can fully register the data output. This will
entail a very high sampling rate and a appropriately sensitive load cell. The best contender for data acquisition is the National Instruments USB-6008. It is capable of
a 48000 sampling rate. But they are slightly expensive, usually falling between \$150 to \$200. This DAQ is easy to setup with MATLAB and can easily and effectively pull the data and
store the data.

\subsection{Faculty Support}
Faculty support will be necessary for obtaining the necessary safety equipment and propulsion equipment, such as the bottle pressures.
There may also be the need for faculty support during the data analysis and nozzle design.

Bottles can be pressurized to 3000 Psi. There is a method of using these bottles for quick propulsion tests. To use these bottles, the team must get faculty help
for safely storing and operating these bottles. The team may also have to get RIT Risk Management's approval for this testing.

\subsection{Long-Term Vision}\label{sec:vision}
The long-term vision of this project is provide experience of analytical nozzle design. Understanding what is happening in the computer that designs the better nozzles, is very important
for propulsion engineers. This project will open a gateway to more complex projects. This project will show the team how to properly contain a high thrust
system. The data analysis experience is also very valueable as it will assist in connecting physical outcomes due to internal conditions. That kind of experience is
invaluable for students in the learning process.

\section*{Acknowledgements}
The author would like to thank Phil Linden and TJ Tarazevits for there continued contribution to the SPEX project definition document process, Jeff Lonneville for
his assistance is providing lab space so SPEX members can conduct propulsion experiments, Dr.Patru for providing storage space, and Dr.Barbosu for his support.

\end{document}
