%% *************************************************************************
%%
%% This is a derivative work of the RIT Space Exploration Standard defining
%% guidelines for content and formatting of project design documents.
%%
%% This document uses IEEEtran.cls, the official IEEE LaTeX class
%% for authors of the Institute of Electrical and Electronics Engineers
%% (IEEE) Transactions journals and conferences.
%%
%% *************************************************************************

%% *************************************************************************
% LaTeX REFERENCES
% ----------------
%   Intro to LaTeX: http://www.rpi.edu/dept/arc/docs/latex/latex-intro.pdf
%   Comprehensive LaTeX symbol list: http://tug.ctan.org/info/symbols/comprehensive/symbols-a4.pdf
%% *************************************************************************

% tell \LaTeX what kind of formatting to use
\documentclass[conference]{IEEEtran} % http://www.ctan.org/pkg/ieeetran
\usepackage{blindtext} % enable placeholder text generator
\usepackage{graphicx} % enable toolbox for embedding figures and pictures
\usepackage{nomencl} % enable package for adding a list of variables and constants at the beginning, aka "nomenclature"
\usepackage{siunitx} % enable package for easily formatting units
\usepackage{hyperref} % enable package for cross-referencing figures, sections, references etc.
% how to use hyperref: http://www2.washjeff.edu/users/rhigginbottom/latex/resources/lecture09.pdf
\usepackage[T1]{fontenc} % change text encoding to make it more crisp
\usepackage{etoolbox} % enable conditionals for help text
\usepackage{booktabs} % make beautiful tables!

% initialize nomenclature package
\makenomenclature{}

% set title. choose something as descriptive and precise as possible. Descriptive > sounding cool. remember this!
\title{CubeSat Launch Initiative Phase I\@: \newline Design \& Research}


\author{
  % List the authors of the design document. The Champion should go first.
  % The \$~\$ markers tell \LaTeX{} to treat the text inside to be treated as a math expression. This way you can use operators like \textcaret{} to place characters as superscripts.
  % Some \LaTeX{} templates handle the author block in different ways. For example, the \href{http://www.worldscientific.com/worldscinet/jai}{Journal of Astronomical Instrumentation} requires the authors' addresses and emails to be included as well.
  % The \textbackslash{}thanks command puts the contents inside those brackets in a footnote at the bottom of the first page. Technically speaking, \textbackslash{}thanks is just a specially formatted footnote.
  % IEEE also has a ``long form'' author block for many authors. Check here for more information:
  % \url{https://tex.stackexchange.com/questions/156523/multiple-authors-with-common-affiliations-in-ieeetran-conference-template}
  % Read here for a more advanced options to modifying footnotes in the author block:  \url{http://tex.stackexchange.com/questions/826/symbols-instead-of-numbers-as-footnote-markers}
  %   Here, we use the IEEE long-form author block.
  \IEEEauthorblockN{% This block is for author Names.
    James~Parkus\IEEEauthorrefmark{1},  %the number in the bracket is a reference number to identify this footnote. \LaTeX will figure out what symbol to put there.
    Austin~Bodzas\IEEEauthorrefmark{2}
  }
  \IEEEauthorblockA{% This block is for the author Affiliations, aka department and university
    RIT Space Exploration, Rochester Institute of Technology \\ %\\ starts a new line
    Rochester, N.Y. \\
    Email:
    \IEEEauthorrefmark{1}jep7631@rit.edu,
    \IEEEauthorrefmark{2}abb6499@rit.edu
  }
  %%   Below, we use the short-form author block and basically hack it to suit our needs.
  % Philip~Linden$^{*\dagger}$%
  %   \thanks{$^{*}$Project Champion}%
  %   \thanks{$^{\dagger}$BS/MEng '17, Mechanical Engineering},
  % Austin~Bodzas$^{\ddagger}$%
  %   \thanks{$^{\ddagger}$BS '19, Computer Science},
  % Drew~Walters$^{\S}$%
  %   \thanks{$^{\S}$BS '18, Mechanical Engineering Technology},
  % T.J.~Tarazevits$^{**}$%
  %   \thanks{$^{**}$BS '19, Game Design \& Development}%

  %%   If there are many authors, consider using symbolic, numeric (aka arabic),  alphabet footnotes or a combination thereof.
  %% the recommended order for symbolic footnotes is
  %%   (1) asterisk        *   *
  %%   (2) dagger          †   \dagger
  %%   (3) double dagger   ‡   \ddagger
  %%   (4) section symbol  §   \S
  %%   et cetera. For higher counts, use 2x symbols (1)-(4) (i.e. (5) two asterisks **). Keep cycling through (1)-(4) using 3x, 4x, and so on.
  %%   Note that these symbol codes work in math mode and text mode.
  %%   There are ways to make LaTeX do this for you, but it is more advanced and not entirely necessary, especially for short author lists. Not worth the hassle, in my opinion.
}
% page header for pages other than cover page
\markboth{CSLI}%
{Parkus \MakeLowercase{\textit{et al.}}: RIT Space Exploration}

% Initial setup is over, start building the document itself
\begin{document}
\maketitle%
% correct bad hyphenation here, separated by spaces
\hyphenation{explor-ation}

\begin{abstract}
The purpose of this project is to go through the design stages of the CSLI\@. The team will choose a scientific payload, engineer the CubeSat to house and function
the payload, begin writing the CSLI by excluding the funding section, and perform a full cost estimation. This stage will leave off with a firm idea of the road ahead for funding and fabrication. The next stage will start with
funding based off the cost estimations this team makes.

      % The abstract is a brief summary of the design document. Typically it includes the purpose of the design document, key goals or objectives, and justifications.
      % Be sure not to confuse the abstract with the introduction.
      % It is easiest to write the abstract after the rest of the paper has been written.
      % That way you can choose key information from the sections that you've already completed and string them together in the abstract.
      % Consider the abstract to be your elevator pitch to anyone reading this design document.
      % What are they reading?
      % What is the goal?
      % Why is it worth my time?
      % The abstract is what will show up in Google results and other search engines, and what people will read when they are deciding what is worth their time and brain power.
\end{abstract}

\label{sec:nomenclature}
\newcommand{\nomunit}[1]{%
\renewcommand{\nomentryend}{\hspace*{\fill}#1}}
\renewcommand{\nompreamble}{
    % If you include mathematical expressions or express variables in the design document, list them with their corresponding definitions here as a list.
    % The two lines below make it look nice when defining units/values to constants.

    % Note that math terms and non-math terms are separated and alphabetized, regardless of the order in which they are defined. (Recall terms \$like this\$ are in the math environment)
    % Read more about advanced nomenclature formatting here:\\
    % \url{https://www.sharelatex.com/learn/Nomenclatures}
  }
\nomenclature{CSLI}{CubeSat Launch Initiative}
\nomenclature{NASA}{National Aeronautics and Space Administration}
\nomenclature{Cal Poly}{California Polytechnic}
\nomenclature{RIT}{Rochester Institute of Technology}
\nomenclature{SPEX}{RIT Space Exploration}
\nomenclature{PDD}{Project Design Document}
\nomenclature{CDR}{Concept Design Review}
\nomenclature{PDR}{Project Design Review}
% Below are examples of using nomenclature for math symbols and constants or units
%\nomenclature{\(\dot{m}\)}{Mass flow rate
%  \nomunit{\,\si{\kilo\gram\per\second}}}
%\nomenclature{\(c\)}{Speed of light
% \nomunit{\,\SI{2.9979e8}{\meter\per\second}}}
%\printnomenclature{}


% HELPFUL HINT: If you get the warning ``Command terminated with space.'' when using a \command try placing ``%'' or ``{}'' immediately following the command.

% The sections included here are required. Additional sections and subsections may be added as necessary.
\section{Introduction}
\label{sec:introduction}
  % The introduction is a place to give background and context before diving into the subject matter.
  % Establish context for the work you are about to propose and the main ideas of the proposition itself.
\IEEEPARstart{T}{he} CubeSat Launch Initiative will be a great opportunity for RIT SPEX members to use their technical skills to make an electromechanical system.
CubeSats were designed with the intent of creating a low-cost opportunity for launching satellites, geared toward allowing university students to put there skills to
real use. The CSLI gives the team a free ride to space, thus eliminating a massive amount of cost to put a satellite in orbit. RIT SPEX will create a team
to design, fabricate, and launch RIT's first CubeSat. There are several phases to this mission and will not be restricted to a tight-timeline. The team will be given ample
time to completely design this system to their maximum potential and perform a full cost estimation. The future stage would be to secure funding. Then to finalize the CSLI
proposal and submit it. If the proposal is accepted the team will begin the fabrication process. The culimination of the fabrication process will be a
fully functional CubeSat ready for experimental and environmental testing.

\section{Primary Objective}
\label{sec:primary-obj}
  % At the end of the day, whether the project ``succeeds'' or ``fails'' is judged against the objectives it sought to meet.
  % Note that results that contradict expectations/hypotheses are not failures if the scientific \& engineering methods are followed along the way.
  % Sometimes our expectations are wrong and that can be just as successful as getting data we thought we'd see.
  % What matters are what questions you intend to answer.
  % This is the main purpose or main goal the project hopes to achieve.
This part of the project, Phase I, will concern itself with the design and research of the CubeSat. This spans everything from picking/researching a payload
to designing the integration of that payload to the CubeSat, with all the features required by the CSLI\@. The primary objective of this phase is to create a
fully developed CAD model, which includes a fully researched and developed scientific payload, and the proper documentation concerned with the electronic features such that the team can write a fully detailed report on
what the CubeSat does and how it does it. When these questions are answered, the next phase of the project will begin.

% \section{Secondary Objectives}
% \label{sec:secondary-obj}
% Secondary Objectives are lower priority or bonus objectives that are significant but not the main focus of the project. This template does not have secondary objectives.

%\autoref{tab:long-example} lists a the relative level of detail expected of the documents written at each stage of a project's life.

%\begin{table*}
% this table is too wide for the two-column format, so we let it expand across both columns
% we haven't told LaTeX where to put this so it'll find the best place.
    %\caption{Relative detail expected at each stage of project development.}
    %\centering
    %\begin{tabular}{@{}llcc@{}}
        % READ THIS!! https://www.inf.ethz.ch/personal/markusp/teaching/guides/guide-tables.pdf
        %\toprule % line on top external edge of table
        % Separate cells in a row with &, move to the next row with \\
      %  Document & Purpose & Contributors & Destination \\
        %\midrule % line separating two internal rows

        % LaTeX doesn't really like multi-line cell contents. Try to keep the text in each cell concise!
      %  \bottomrule
  %  \end{tabular}
%\label{tab:long-example}
%\end{table*}

\section{Benefit to SPEX}
\label{sec:benefit}
% One of the core values of SPEX is to provide opportunities for academic and professional growth for its members,
% and to challenge them with interesting projects.
% In this section, explain how the project would benefit SPEX members as students,
% space enthusiasts, and young professionals.
This project would be beneficial to SPEX for the following reasons; experience (project management, engineering, coding, etc.) given to current members, opportunities
for data analytics on collected data, PR, provides base for further funding and sponserships.
The PR for this would be great for getting recognition in the aerospace community. A successful CubeSat is
difficult to attain. If SPEX is able to excute this project well, it would be a serious step in being national recognized as an aerospace/astronautical research group.
The marketing team at RIT SPEX would be able to leverage this project to get interested companies to donate supplies, money, or their time so RIT SPEX can build
and grow in their research. The experience for the team would be great because it would bring a project from pure design and research to flight. Allowing students
to be part of that project lifecycle would be tremendous for growth in their respective field.

\section{Implementation}
\label{sec:implementation}
  % What path do you anticipate the project to take?
This project will start with a small design and research team. That team will be responsible for picking and appropriate payload. Once the payload is chosen, the
design of the CubeSat will begin. The team will bring in members that have experience in the relevant categories, such as radio communication, structural engineering,
software and firmware development, etc. The team will be responsible for creating thorough documentation on every aspect of the CubeSat to ensure future members will be able to pick it up and
keep going without unnecessary time-loss and scope-creep. The team will strive to attend as many RIT poster sessions as possible, once ready, to
build recognition of the project, its goals, and its status. That is imperative for when the funding phase arrives.

The project will follow \textit{the Need-Driven Process}. This involves following a need-based mission. The steps involved in developing this mission are listed in \autoref{tab:need driven process}.

\begin{table}[h!]
    \caption{Steps for the Need-Driven Process}
    \centering
    \begin{tabular}{@{}cl@{}}
    \toprule
    Step & Description \\
    \midrule
    1 & Define mission needs \\
    2 & Identify principal players \\
    3 & Define timeline over which the program needs to be completed \\
    4 & Quantify mission details \\
    5 & Define alternate combinations of mission elements \\
    6 & Develop alternative mission concepts \\
    7 & Critical requirements \\
    8 & Performance Assessments and System trades \\
    9 & Quantify how well the broad objectives \\
    10 & Creating a baseline design \\
    11 & Revise system requirements and constraints \\
    12 & Design iterations \\
    13 & Begin traditional systems engineering process \\
    14 & Flow down numerical requirements \\
    \bottomrule
    \end{tabular}
\label{tab:need driven process}
\end{table}

Step 1 of the process is to define the needs that the mission must achieve. What are the quantitative goals, and why? This information should come
from a mission statement of what the mission is attempting to achieve. Step 2 identifies the prinicipal players (aka stakeholders) and the space community
of which they are a part. Step 3 defines the timeline over which the project needs to be excuted to be useful. Step 4 quantifies how well the team wishes
to achieve the broad objective, given the team needs, applicable technology, who the users are, and cost and schedule constraints. These requirements are
flexible and subject to change throughout the project. Step 5 defines the alternate combinations of mission elements or the \textit{space mission architecture} to meet
the mission objectives and requirements. Step 6 develops alternative mission concepts. In step 7 we identify the prinicpal cost and performance drivers for each
alternative mission concept. In step 8 the team will conduct performance assessments and system trades. It will define in detail what the system is and does. Step 9
quantifies how well we are meeting both the broad objectives and the needs of the end user as a function of either cost or key system design choices. In step 10 the team
will select one or more baseline system designs. In step 11 the team revises the system requirements and constraints consistent with what the team has learned, and in step 12
the team will explore other alternatives and iterate upon the design. The team will translate the now better-defined objectives, constraints, and requirements into well-defined
system requirements in step 13. Finally in step 14 the team flows down these numerical requirements to the components of the overall space mission.

These steps are derived from \textit{Space Mission Engineering: The New SMAD}, from the space mission engineering process chapter concerning the need-driven process for projects.
\subsection{Deliverables}
\label{subsec:deliverables}
  % When all is said and done, what will you have to show for it?
  % Examples: Hardware, software, poster, ImagineRIT demo, presentations, technical papers...
The team will make 2 or 3 postings on the RIT SPEX website, per semester, with progress and science updates to keep the project in outside contributors/persons of
interest's field of view. The team will be required to have a poster for Imagine RIT and the undergraduate research symposium in the spring and summer terms, respectively.

The documentation for a project of this size and length is essential. Since the project is going to take a few semesters, at minimum, to complete. It is important
to have enough documentation on each aspect to be able to bring new members up-to-speed as easy and quick as possible. The documentation should cover the following; fully end-to-end CubeSat assembly,
necessary hardware, a full drawing packet on the CubeSat design, design requirements, engineering requirements, payload functionality, payload description, payload integration specifications, PCB design,
PCB layouts, PCB components, power requirements and distribution diagram, radio signal mapping, and FEA simulation analysis. These documents should be saved in individual locations
and concatenated into a master document, for easy reference.

There will be weekly meetings to discuss progress on different aspects of the project being investigated by different persons. Each member will be expected to discuss
their progress and contribute to a master logbook in which documentation will be recorded for future reference. The documents can range from code to articles on specific calculations
or designs, etc. Powerpoints are preferred for presenting to other members. The team will use the team communication software called WebEx by Cisco for teleconferencing and telecommuncations.

This phase of the project will be considered complete when the following is delivered; a fully developed CAD model, electrical diagrams describing the electrical functionality,
 a full fabrication estimation (this does not include an cost for experimentation), and the CSLI proposal drafted (excluding the funding section).

\subsection{Milestones}
\label{subsec:milestones}
  % Be as detailed as you can, but it's okay if there are unknowns.
  % At the very least, specify how many semester you expect the project to take until it reaches completion.
The milestones for this project include surpassing each step laid out in \autoref{sec:implementation}. The more general project milestones are choosing a payload,
designing the payload, CubeSat structure, CubeSat avionics and software. Then also creating a risk assessment based on the first design iteration.

Once these first milestones are completed the team will go through a concept design review. This will give them early feedback based on their design decisions and
allow them the opportunity to address concerns before the system design is frozen. Then the team will iterate upon their design further, figure out more details and
go back for a project design review with another risk assessment. After the project design review the team will make final design iterations and prepare for the funding phase.
Preparation for funding will include making many documents and graphics to show the project to potential investors and or to please already dedicated investors.

\section{Externalities}
  % Things not directly related to the work or outcomes, but related to the project as a whole.
\subsection{Prerequisite Skills}
  % Which skills do team members need to have before work can start (not including skills that will be learned ``on the job'')?
The team will have a maximum of 7 members during the design and research phase. At minimum, the team will need 3 mechanical engineers, 1 electrical engineer, and 1 computer scientist (or avionics specialist).
It would be most beneficial to have mechanical, electrical, software engineers, physicst, and a computer science major. The team members will be responsible for fulfilling roles
that may extend beyond the reach of their speciality. While it extends beyond their speciality, it will not extend beyond their capability. In the event that is unavoidable, outside
help will be sought through professors or the appropriate student body.

In general, it is expected that mechanical engineers will be profficient in Solidworks and MATLAB. Solidworks is the chosen 3D modeling software due wide availability and experience.
MATLAB will be used for calculating orbits or performing mechanical analysis. It is not required for people that have a refined coding ability. Mechanical engineers will need to
have taken a strength of materials, thermodynamics, and the university physics sequence, or equivalent courses. They will also be responsible for performing finite element analysis
on the system. This requires previous knowledge on the subject.

Electrical engineers may be responsible for designing circuit board layouts, coding the boards, or figuring out the electrical requirements for the payload in general.

\subsection{Funding Requirements}
  % Estimate costs that would be needed to meet objectives.
The funding requirements for this design phase are minimal. There may be some cost during feasability studies if a particular technology must be demonstrated.
But that will likely be low-cost. The official cost estimation for this phase of the project will be \$300. All the programs the team will need, ANSYS, Solidworks, KiCAD, etc., will
be provided by the university or by sponsers.

\subsection{Faculty Support}
  % Identify faculty that will be involved (or would need to be involved) to meet objectives.
  % Note that if a professor is the Principal Investigator (P.I.) for a project, there still needs to be a student as the SPEX Project Champion.
Faculty support is difficult to gauge based on the ambiguity of the payload. Since the payload is completely unknown at this time, the extent of faculty support will be considered when
the payload is selected.

\subsection{Long-Term Vision}
\label{sec:vision}
The long-term vision of this project is to launch a CubeSat. This project begins with the most important phase, the initial design and research. Once this is complete the team will move onto
funding. Then it will complete the CSLI proposal and submit it once there is a call for papers for a CubeSat mission. If the project is selected, the team will begin fabrication. The secured funds will
be used to purchase the necessary components to test the system at component-level then system-level. Once the system is complete, it will be flight-prepared and launched when possible. Once this project nears the
final design of the payload, the funding phase should start. Since the payload will mostly be known at that time, it would be beneficial to start getting in contact with companies to get a foot in the door while this phase is
finalized.

\begin{table}[h!]
    % the "h" in these brackets tells LaTeX to put the table Here. Try [t] for top and [b] for bottom,
    % or [hbp] for "here, or if you can't do that put it at the bottom of the page, or if you can't do that put it on its own page.
    % Here we've also used an "!" to yell at LaTeX to DO THIS OR ELSE!
    \caption{CSLI Phases}
    \centering
    \begin{tabular}{@{}cll@{}}
    % the letters here ^^^^ designate the columns.
    % (l=left align, c=center, r=right align)
    % the weird @{} thingies tell LaTeX to not have left-right padding between cells
    % so cells butt up right against the edge
    \toprule
    Phase \# & Title & Purpose \\
    \midrule
    1 & Design \& Research & Chose payload, design CubeSat, \\
      &                    & write CSLI (without funding section) \\
    2 & Funding & Secure funding \\
    3 & CSLI Submission & Get CSLI proposal reviewed and submitted \\
    4 & Build & Build CubeSat with Payload \\
    5 & Testing & Perform system-level testing \\
    6 & Launch & Launch the CubeSat \\
    7 & Science & Retrieve data from CubeSat and payload \\
    \bottomrule
    \end{tabular}
\label{tab:phases}
\end{table}

The team that works on this project does not need to be on the future stages of the project. The goal of the rigorous documentation is to allow any and all members
to pick up where this team left off, with minimal reading to get caught up to speed.
\section*{Acknowledgements}
The author would like to thank Dr.~Bill Destler and Rebecca Johnson for being exemplary humans, Anthony Hennig for founding RIT Space Exploration, and all the SPEX members that continue to invest their time and energy into the pursuit of space exploration.

%\bibliographystyle{IEEEtran}
%\bibliography{sample-with-examples}

\onecolumn
\appendices{}
\end{document}
