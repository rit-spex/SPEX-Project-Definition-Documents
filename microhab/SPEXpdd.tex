%% *************************************************************************
%%
%% This is an RIT Space Exploration Standard defining guidelines for content
%% and formatting of project design documents.
%%
%% This document uses IEEEtran.cls, the official IEEE LaTeX class
%% for authors of the Institute of Electrical and Electronics Engineers
%% (IEEE) Transactions journals and conferences.
%%
%% *************************************************************************

%% *************************************************************************
% LaTeX REFERENCES
% ----------------
%   Intro to LaTeX: http://www.rpi.edu/dept/arc/docs/latex/latex-intro.pdf
%   Comprehensive LaTeX symbol list: http://tug.ctan.org/info/symbols/comprehensive/symbols-a4.pdf
%% *************************************************************************

\documentclass[conference]{IEEEtran} % http://www.ctan.org/pkg/ieeetran
\usepackage[euler]{textgreek}
\usepackage{blindtext}
\usepackage{graphicx}
\usepackage{nomencl}
\usepackage{siunitx}
\usepackage{hyperref}
\usepackage[T1]{fontenc}
\usepackage{etoolbox}
\makenomenclature{}

\title{\textmu{}Hab}


\author {
  \IEEEauthorblockN{% This block is for author Names.
    Austin~Bodzas\IEEEauthorrefmark{1}
  }
  \IEEEauthorblockA{% This block is for the author Affiliations, aka department and university
    RIT Space Exploration, Rochester Institute of Technology \\ %\\ starts a new line
    Rochester, N.Y. \\
    Email:
    \IEEEauthorrefmark{1}abb6499@rit.edu
  }
}
% page header for pages other than cover page
\markboth{\textmu{}hab}%
{Bodzas \MakeLowercase{\textit{et al.}}: RIT Space Exploration}

% Initial setup is over, start building the document itself
\begin{document}
\maketitle%
% correct bad hyphenation here, separated by spaces
\hyphenation{explor-ation}

\begin{abstract}

      % The abstract is a brief summary of the design document. Typically it includes the purpose of the design document, key goals or objectives, and justifications.
      % Be sure not to confuse the abstract with the introduction.
      % It is easiest to write the abstract after the rest of the paper has been written.
      % That way you can choose key information from the sections that you've already completed and string them together in the abstract.
      % Consider the abstract to be your elevator pitch to anyone reading this design document.
      % What are they reading?
      % What is the goal?
      % Why is it worth my time?
      % The abstract is what will show up in Google results and other search engines, and what people will read when they are deciding what is worth their time and brain power.
    A project definition for the \textmu{}Hab project.  \textmu{}Hab is the term the
    HAB team has been using to describe a smaller balloon and payload than is
    typically used by the group (1200 grams). However, this project expands upon
    the term with the main goal of minimizing costs that go into building and
    launching a HAB.\@
\end{abstract}

\label{sec:nomenclature}
\newcommand{\nomunit}[1]{%
\renewcommand{\nomentryend}{\hspace*{\fill}#1}}
\renewcommand{\nompreamble}{}
\nomenclature{RIT}{Rochester Institute of Technology}
\nomenclature{SPEX}{RIT Space Exploration}
\nomenclature{PDD}{Project Design Document}
% Below are examples of using nomenclature for math symbols and constants or units
% \nomenclature{$\dot{m}$}{Mass flow rate
%   \nomunit{\,\si{\kilo\gram\per\second}}}
% \nomenclature{$c$}{Speed of light
%  \nomunit{\,\SI{2.9979e8}{\meter\per\second}}}
\printnomenclature{}


% HELPFUL HINT: If you get the warning ``Command terminated with space.'' when using a \command try placing ``%'' or ``{}'' immediately following the command.

% The sections included here are required. Additional sections and subsections may be added as necessary.
\section{Introduction}
\label{sec:introduction}
  % The introduction is a place to give background and context before diving into the subject matter.
  % Establish context for the work you are about to propose and the main ideas of the proposition itself.

\IEEEPARstart{S}{pex} has launched 3 High Altitude Balloons (HABs) to date.  The
motivate for a launch is to educate students on engineering, test experiments in
a near space-like environment, and to have fun. The HAB team within Spex aims to
launch a balloon at the end of every full semester.

Traditional HAB launches are very expensive for the SPEX team.
A large 1200g balloon can cost, at a minimum, over 125USD.  Helium tank fills,
which these launches require a full fill, can cost over 200USD. These two costs
are a definite non-reusable resource. An avionics package consisting of COTS
Arduino, breakout sensors, and a large battery pack can easily add up to over
90USD. The avionics hardware can be reused but every launch should be assumed a
one time only use of hardware due to high risks of losing the payload.


\section{Primary Objective}
\label{sec:primary-obj}
  % At the end of the day, whether the project ``succeeds'' or ``fails'' is judged against the objectives it sought to meet.
  % Note that results that contradict expectations/hypotheses are not failures if the scientific \& engineering methods are followed along the way.
  % Sometimes our expectations are wrong and that can be just as successful as getting data we thought we'd see.
  % What matters are what questions you intend to answer.
  % This is the main purpose or main goal the project hopes to achieve.
\textmu{}Hab will reduce the magnitude of these three large costs that make frequent HAB
launches unaffordable for undergraduate research. A smaller, simpler, more
focused payload will result in cheaper avionics hardware costs. A lighter
payload also doesn't require such a large balloon, enabling a cheaper, smaller
balloon and orders of magnitude less helium use per launch.

\section{Secondary Objectives}
\label{sec:secondary-obj}
% Secondary Objectives are lower priority or bonus objectives that are significant but not the main focus of the project. This template does not have secondary objectives.
Educational benefit is a very large secondary objective of \textmu{}Hab.  This
project shall be designed to be engineered in a fashion that teaches both
established senior members and new greenhorn SPEX members alike.

Electrical hardware design is a skill that many members have shown a desire to
pick up. Design of the PCB shall be simplistic and incorporate a few different
important skills for engineering in the electrical hardware domain.

Embedded software is another useful skill that many students in the Computing
skill aren't taught formally. Arduino development and embedded C software
engineering skills will be obtained from writing code for the avionics board.

\section{Benefit to SPEX}
\label{sec:benefit}
% One of the core values of SPEX is to provide opportunities for academic and professional growth for its members,
% and to challenge them with interesting projects.
% In this section, explain how the project would benefit SPEX members as students,
% space enthusiasts, and young professionals.

Each of the objectives grant their own benefit to SPEX.\@ Cheaper launch costs
allow for more SPEX funding to be allocated to other projects while not
sacrificing the quality of HAB launches significantly.

Educational benefit, while not the primary benefit, could be the largest
benefit.  PCB design has always been a very in demand with low supply skill.
Embedded software is required for almost every large project. Arduino software
design is a great stepping stone for software oriented people into the world of
embedded. Avionics design in C would be the final transition, from Arduino into
the real world application of embedded software.

% Below I have used subsections to identify key ideas in this section. These particular subsections are not required as part of the SPEX Standard, but serve as an example of using subsections in a text.

\subsection{Accessibility}
\label{subsec:plug-n-play}
  % Note below that LaTeX uses weird formatting when it comes to quotation marks.
  % The style below is correct to display forward quotes `` at the start of the phrase and backquotes '' at the end.

In regards to the software half, Arduino is a very approachable language.  There
is a vast amount of libraries out there already for interfacing with ancillary
hardware such as sensors or memory. The C portion to come at a later date (or in
parallel) will be quite a bit more difficult; it requires a ton of datasheet
reading, innovative code (not taking from examples), and customizing toolchains.

PCB design accessibility is a lot more restricted. Members that have had PCB
design classes (4th-5th year Electrical Engineers) or have experience from co-op
will be able to contribute to the design.



\section{Implementation}
\label{sec:implementation}
  % What path do you anticipate the project to take?

In the ideal case, every project begins with a design document.
That design document gets sent around to SPEX members (and non-members) to draw support and build a team.
Research and work takes place, documented along the way until  an ending point is reached (e.g.\ project completion, end of the semester, team attrition, etc.).

At the end of the project (or end of semester, whichever comes first), the team writes a report of the project with what they did, if it was successful, and recommendations for future projects.
A future SPEX member might pick up where the last paper left off, and the cycle repeats.

\subsection{Deliverables}
\label{subsec:deliverables}
  % When all is said and done, what will you have to show for it?
  % Examples: Hardware, software, poster, ImagineRIT demo, presentations, technical papers...
\begin{itemize}
    \item Bill of Materials for Hardware
    \item PCB schematics
    \item Arduino software source files
    \item C software source files
    \item Fully assembled balloon, parachute, and payload
\end{itemize}

\subsection{Milestones}
\label{subsec:milestones}
  % Be as detailed as you can, but it's okay if there are unknowns.
  % At the very least, specify how many semester you expect the project to take until it reaches completion.
\subsubsection*{Electrical}
\begin{itemize}
    \item 2 weeks --- PCB design
    \item 1 week --- PCB reviews
    \item 1 week --- Order hardware
    \item 1 week --- PCB assembly
    \item 1 week --- PCB testing
\end{itemize}

\subsubsection*{Arduino software}
\begin{itemize}
    \item 1 week --- gather libraries needed
    \item 1 week --- plan out software design and flow
    \item 2 weeks --- write software
    \item 1 week --- test software
    \item 1 week --- neaten up software
\end{itemize}

\subsubsection*{C software}
\begin{itemize}
    \item 1 week - prepare standard development environment
    \item 1 week - plan out software design and flow
    \item 4 weeks - create drivers
    \item 8 weeks - write software
    \item 2 weeks - test software
    \item 1 week - neaten up software
\end{itemize}
\section{Externalities}
  % Things not directly related to the work or outcomes, but related to the project as a whole.
\subsection{Prerequisite Skills}
  % Which skills do team members need to have before work can start (not including skills that will be learned ``on the job'')?
It is obvious that team members will learn certain skills as a project progresses, but there are always some tasks that require a minimum skill level to provide meaningful contributions to a project's development.
These prerequisite skills are best identified by examining past projects and discussing the project with faculty or subject matter experts.
It is strongly recommended to be conservative in skill estimation.
Underestimate team member skill levels and overestimate the challenge.
Many projects have failed because the team overestimated their own abilities or underestimated the difficulty of their project.

\subsection{Funding Requirements}
  % Estimate costs that would be needed to meet objectives.
Like prerequisite skills, it is wise to overestimate the cost of components, materials and other resources that a project requires.
For physical projects, costs may be estimated by benchmarking the costs of similar systems or determining a representative bill of materials and using the aggregate cost of its items.

\subsection{Faculty Support}
  % Identify faculty that will be involved (or would need to be involved) to meet objectives.
  % Note that if a professor is the Principal Investigator (P.I.) for a project, there still needs to be a student as the SPEX Project Champion.
Support from university faculty is almost always essential to a project's success.
Faculty provide not only guidance and subject matter expertise, but may also connect a team with resources and networking opportunities.
SPEX projects do not require faculty support, but it is highly recommended to identify professors with an interest or expertise in a project as early as possible.

\subsection{Long-Term Vision}
\label{sec:vision}
As SPEX student members get more experience writing these papers, the group will build a library of meaningful work and be able to save it in an organized manner.
Knowledge will be preserved and easily shared.
Perhaps Project Design Document could eventually get published, in a journal or otherwise\ldots

\section*{Acknowledgements}
The author would like to thank Dr.~Bill Destler and Rebecca Johnson for being exemplary humans, Anthony Hennig for founding RIT Space Exploration, and all the SPEX members that continue to invest their time and energy into the pursuit of space exploration.

\onecolumn
\appendices{}
\section{Project Life Cycle}

\end{document}
