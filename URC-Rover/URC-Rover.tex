%% *************************************************************************
%%
%% This is the PDD for Rovers 2018
%%
%% This document uses IEEEtran.cls, the official IEEE LaTeX class
%% for authors of the Institute of Electrical and Electronics Engineers
%% (IEEE) Transactions journals and conferences.
%%
%% *************************************************************************

\documentclass[conference]{IEEEtran} % http://www.ctan.org/pkg/ieeetran
 \usepackage{ltxtable, tabularx, longtable} 
\usepackage{tabularx} % fix tables
  \newcolumntype{L}{>{\raggedright\arraybackslash}X}   
\usepackage{multirow}  
\usepackage{blindtext} % enable placeholder text generator
\usepackage{graphicx} % enable toolbox for embedding figures and pictures
\usepackage{nomencl} % enable package for adding a list of variables and constants at the beginning, aka "nomenclature"
\usepackage{siunitx} % enable package for easily formatting units
\usepackage{hyperref} % enable package for cross-referencing figures, sections, references etc.
% how to use hyperref: http://www2.washjeff.edu/users/rhigginbottom/latex/resources/lecture09.pdf
\usepackage[T1]{fontenc} % change text encoding to make it more crisp
\usepackage{etoolbox} % enable conditionals for help text
\usepackage{booktabs} % make beautiful tables!
\usepackage[utf8]{inputenc}
\usepackage[english]{babel}
 
\usepackage{hyperref}
\hypersetup{
    colorlinks=true,
    linkcolor=blue,
    filecolor=magenta,      
    urlcolor=cyan,
}

% initialize nomenclature package
\makenomenclature{}

% set title. choose something as descriptive and precise as possible. Descriptive > sounding cool. remember this!
\title{University Rover Challenge Ready Rover} %TODO come up with a better title

\author{
  \IEEEauthorblockN{Thomas~Hall\IEEEauthorrefmark{1}
  Carter~Miller\IEEEauthorrefmark{2}
  Clayton~Terryl\IEEEauthorrefmark{3}
  Ben~Hebert\IEEEauthorrefmark{4}
  Alex~Olds\IEEEauthorrefmark{5}}
  \IEEEauthorblockA{RIT Space Exploration, Rochester Institute of Technology \\ Rochester, N.Y. \\ Email: \IEEEauthorrefmark{1}tjh2822@g.rit.edu \IEEEauthorrefmark{2}cjm3473@g.rit.edu \IEEEauthorrefmark{3}cjt4285@g.rit.edu \IEEEauthorrefmark{4}bbh8195@g.rit.edu \IEEEauthorrefmark{5}apo6377@g.rit.edu}
}

% page header for pages other than cover page
\markboth{Project Design Document Standard}%
{Hall \MakeLowercase{\textit{et al.}}: RIT Space Exploration}

\begin{document}
\maketitle
\hyphenation{explor-ation}

\begin{abstract}
This Project Definition Document serves as an overview to, and provides insight into the technological aspects of the RIT Space Exploration Mars Rover. The Mars Rover is a robot built to compete in the 2021 University Rover Challenge (URC), hosted by the Mars Society. The competition is held annually but we are planning on taking 2 years to build and test the rover. This robot is RIT’s 1st entry to the competition. At a glance, this project may appear to be a simple expansion of RIT Space Exploration’s mini rover. However, URC Mars Rover is the result of a much more ambitious effort in engineering as almost all of the components are designed from the ground up by team members. The URC ready rover is substantially more capable than the mini rover. While upgrading the suspension to be Utah desert ready will be tough, designing and building the robotic arm is a challenging project in itself.  
Index Terms—University Rover Challenge; URC; The Mars Society; Mars rover; robot; mini rover
\end{abstract}

\label{sec:nomenclature}
\newcommand{\nomunit}[1]{\renewcommand{\nomentryend}{\hspace*{\fill}#1}}
\renewcommand{\nompreamble}{}

\nomenclature{RIT}{Rochester Institute of Technology}
\nomenclature{SPEX}{RIT Space Exploration}
\nomenclature{PDD}{Project Design Document}
\nomenclature{URC}{University Robotics Competition}
\nomenclature{MVP}{Minimum Viable Product}

% Intro
\section{Introduction}
\label{sec:introduction}

This report is an overview and insight into the technical details of the 
2021 RIT Space Exploration (RIT SPEX) Mars Rover, a robot built to compete in the 2021 
University Rover Challenge hosted by the Mars Society. This robot is RIT SPEX’s first entry into the c
ompetition. The competition takes place each year at the Mars Desert Research Station 
managed by the Mars Society, located near Hanksville, Utah, USA. The dates for the 2021 
competition have yet to be finalized but they are usually around the first weekend in 
June and last 4 days or so. All tasks must be completed in a certain time limit, 
usually ranging between 30-45 minutes. All operators of the rover must not be 
able to view the rover directly. All control must be performed using onboard 
cameras and/or sensors. The weight and value of the robot is limited to 50 kg
(110 lbs) and \$18,000.

\section{Primary Objective}
\label{sec:primary-obj}
The primary objective of this project is to design and build a rover to compete in the 2021 University Rovers Challenge (URC), hosted by the Mars Society. To compete, we bring together a team of diverse and motivated students to design, build, and operate a sophisticated mock Mars Rover. We will be designing all aspects of the rover ourselves as well as doing much of the machining for rover ourselves. 

\section{Secondary Objectives}
\label{sec:secondary-obj}
 There are four scenarios in which the robots will be required to perform varying tasks, they are as follows. There is also a presentation to the judges
 
Site Survey: Perform a remote survey to determine the precise coordinates (including altitude) of field markers that are unreachable by the rover.
 
Sample Return: Collect and return samples from sites determined to have the greatest likelihood of containing photosynthetic bacteria, other bacterial colonies, and nonbacterial extremophiles such as lichen.
 
Equipment Servicing: Perform several dexterous operations, such as pushing buttons, flipping switches, and connecting plugs into electrical outlets, on an equipment panel at a remote location according to the instructions printed on the panel.
 
Astronaut Assistance: The rover must locate and distribute packages weighing up to 6 kg (13.2 lbs) from the rover to up to five astronauts working in the environment of Mars.
 
Presentation: Give a 15 minute presentation describing the overall design of the rover, team structure and project budget, followed by a short question and answer session with the judges

The project will also be submitting the rover to the University Rover Challenge (URC) as hosted by the Mars Society. The competition is held annually and features four very intense competitions. Such a project would be among the most ambitious SPEX has ever attempted. Second to the CubeSat at least. The team will accomplish this by taking a 2 year build society as well as breaking the team into 4 sub-teams. By delegating the responsibilities among members this project can go far.

\section{Benefit to SPEX}
\label{sec:benefit}

\subsection{Space Exploration}
\label{subsec:spacex}
Rovers are a huge part of space exploration. There are eight active missions on Mars. Four of those include rovers. To ignore this fact would be to ignore a large part of space exploration. Currently RIT SPEX is not involved with rovers.It would be beneficial to our members to get some experience in this area.

\subsection{Promotions and Funding}
\label{subsec:promos}
A rover would be a popular demonstration at ImagineRIT. The rover would be rather large and would attract many eyes. We could even demo it outside if there is sufficient space. A rover would be very easy to get video and photos of for SPEX promotions. Having a rover is also another opportunity for SPEX to fundraise. There is plenty of space to place company logos on the body of the rover. It would also allow for SPEX to reach out to robotics companies.

\subsection{Computer Science}
\label{subsec:cs}
The potential self-driving component would be the heaviest computer science project SPEX would have attempted, this would help with retention of CS and SE majors. There are many features included in the MVP that require significant computer science skills as well. This includes navigation, power management, communications and more. That will give computing-related majors another option for SPEX involvement, also helping retention.
\subsubsection{Artificial Intelligence}

\subsection{Artificial Intelligence}
\label{subsec:ai}
Some of the potential self-driving features will likely be implemented with artificial intelligence. In particular the fully autonomous navigation would require this. Machine learning and artificial intelligence are at the forefront of computer science right now. They are heavily desired in industry, including space exploration. There are limited options for RIT students to get involved with AI project wise, this can help with that. There is another AI-based student group (RITficial Intelligence) on campus that has members interested in these projects but the group lacks the funding to start any. RIT SPEX could partner with this group to recruit talent.

\subsection{Robotics}
\label{subsec:robotics}
Robotics are also a heavily desired area in the space industry. Currently RIT SPEX does not have any projects with robotics as a focus. Robotics is also a potential area for collaboration with faculty as there are RIT faculty that are investigating this area. RIT is host to the FIRST robotics competition which would be a good source to get information and help from. However, most university students volunteer to run this competition instead of participating. This competition is at the high school level and many students want an opportunity to do something more advanced. SPEX could even show off the rover at the FIRST robotics competition (especially if the rover competed in the University Robotics Competition or a similar competition). RIT would love the promotional value it would be a great recruitment and fundraising opportunity for RIT SPEX.

\subsection{Challenge}
\label{subsec:challenge}
The project including just a few of the improvements would be very ambitious. It would require at a minimum a handful (4-5) members each with different areas of expertise just for the MVP. The team would have to be even larger to accommodate some of the improvements as a larger team would be able to accomplish more. Aside from technical skills, members would need to be knowledge in LaTeX, documentation, notetaking, and good research practices. Challenging projects allow for members to gain even more experience. 

\subsection{Documentation}
\label{subsec:docs}
It is important that the team members document the sources they use to gather information. There will be a Google Drive folder that will hold notes with links to any books, articles, media or their sources that are relevant to the rover project. The software and hardware will be tracked on GitHub (including the technical report). One of the areas SPEX is currently struggling with is knowledge transfer. This project should serve as an example to other SPEX projects on how information should be passed on from year to year. Being able to successfully transfer knowledge from year to year will greatly advance the ability for RIT SPEX to complete more ambitious projects.

\subsection{Project Management Skills}
\label{subsec:project-management}
The purpose of this project is to investigate and implement a rover. Because of this the team members must be in the mindset to analyze each part of the project and identify as many problem areas that need answers as possible. This means being specific on how we are going to accomplish our goals. We want our team asking the question:"what material, what algorithm, with what method will we be accomplishing this?" It is very easy to overestimate the team's ability, especially when the project is so new to the team. These are skills that benefit RIT SPEX.

\section{Implementation}
\label{sec:implementation}

\subsection{Deliverables}
\label{subsec:deliverables}
The primary deliverable of this project is the rover itself. As per the competition rules the team must also subject itself to two design reviews in the form of prepared videos. The team will be delivering these videos as well as the required paperwork and documentation that the competition requires. 

\subsubsection{Milestones}
\label{milestones}
This project will be relatively long as we are looking to compete for the first time in May of 2021, meaning we have four semesters and one summer term to complete the URC rover. 
Timeline:
Registration/Intent to Compete: 
Early November 2020
Preliminary Design Review (PDR)
Late November 2020
System Acceptance Review (SAR) (20% of final URC score)
Early March 2021
Declare Comms Standards
Late April 2021
Written Science Plan
Early May 2021
Budget Information Submission
Mid May 2021
Competition
Late May 2021

\subsubsection{Rover Design}
\label{roverdesign}
The mini rover serves an  a successful  proof of concept. It is key to the development of the URC rover. The prominent design features of the URC Mars Rover include;

1) Four with balloon tires: Placing the rover on four balloon tires distributes the weight over a large area, making it easy to drive in sand and improving skid-steer performance. In addition, the wheel spacing reduces the risk of jamming objects between the wheels. 

2) Direct drive: Each wheel is mounted to its own dedicated drive motor, eliminating the need for additional shafts, sprockets, chains, and bearings, reducing overall weight and complexity. Additionally, placing the motors inside the wheels locates some of the heaviest components very close to the ground. This lowers its center of gravity, thereby improving the rover’s ability to navigate steep terrain without the risk of toppling over. 

3) Flexible chassis with high ground clearance: The chassis reduces the chance of becoming high-centered on obstacles and insures all four wheels stay on the ground, greatly improving all-terrain performance. This type of chassis does not require any springs, bearings, or shock absorbers, reducing complexity and improving reliability. 

4) Non-skid-steer and zero radius turning ability: By being able turn in place without skidding greatly increases the Rovers all-terrain capability as it no longer requires smooth or loose surfaces to turn on; it is capable of turning over large, complex obstacles. 

5) Ergonomic electronics bay: If the electronics require troubleshooting or repair, the entire electronics bay can be removed from the chassis with a few wing nuts. This allows the electrical and mechanical teams to simultaneously work on different subsystems in separate locations. For example, machining work can be performed on the chassis while the electronics bay is in the lab for testing of electrical systems. The electronics bay could be accessed by popping two latches and removing the cover. This saved a lot of time during testing, this allowed for testing of Rover subsystems without their removal. Additionally, each of the custom designed electronic modules interfaced with the rover via a back plane. If a module required testing or fixing, it could simply be pulled out of its slot without the need of tools. Modules could also be reinstalled without the possibility of incorrect installation as they can be installed in any back plane slot. 

6) Adjustable camera mast: The use of a tripod allows the camera to be quickly located anywhere over the rover, providing any point of view desirable. 

7) Weight: The rover was marginally under the weight limit of 50 kg (110 lbs) during competition weigh-in. the team could have incurred a penalty if they had been unable to make sacrifices to make weight. It is recommended that the rover be obviously under-weight as to reduce any potential for penalty. 

8) 3D vision system: A 3D vision system is an interesting and potentially advantageous system to have on the rover and it should not be discouraged from being developed, but it is not essential to the robots minimal functionality. It could have potential use in future designs.

\section{Externalities}
\label{sec:externalities}
\subsection{Prerequisite Skills}
\label{subsec:skills}
This project will require members will knowledge in mechanical design and manufacturing, software engineering, electrical engineering, chemistry as it relates to life on other planets and general project management and leadership. Skills in research and documentation will also be required. A number of mechanical, electrical and software engineers are already on the project. However, finding SPEX members with skills in the science required for that portion of the competition may prove to be difficult. Building a rover to URC specifications will require a large team with a diverse set of technical skills. A major first step for this project will include recruiting new members to SPEX, hopefully freshmen who can shadow more skilled members of the team, as well as existing SPEX members who may not yet be on a project.

\subsection{Funding Requirements}
\label{subsec:funding}
Because this project required significant material resources, it is vital to find support in the form of donated products as well as financial support. The miniature rover made last year was mostly funded through SG, SPEX and personal donations on behalf of the members on the project. Creating a competitive URC Rover will be significantly more expensive, due to components such as the LIDAR sensors, mechanical arm and long range remote communications. The maximum allowable budget is \$18,000 for the entire project. This applies to all rover components and equipment, but not spare parts, tools or travel expenses. We may reuse equipment, but it must still be accounted for in records that may be required to be submitted. For most of the required components on the 2021 Mars Rover, vendors were contacted and asked if they would be willing to sponsor the team by donating the required components. In some cases, companies were willing to donate the products and provide additional financial support, covering the costs of non-donated components and logistical expenses. In return for the donations, all sponsors will be receiving publicity proportional to the combined value of their respective donations. This dictates the size and visibility of logos on the rover and banner, as well as how frequently and to what extent the contribution of these sponsors are to be highlighted at public events. Sponsors will also be provided with bi-monthly updates that allow them to follow the team’s progress leading up to competition. Often sponsors might have questions in which case these will be answered as quick and thoroughly as possible. Thu updates will allow for the sponsors to understand better how their donations will be being put to use. The support will be provided by each of the team’s sponsors was vital to the team’s existence, and has been sincerely appreciated.

\subsection{Faculty Support}
\label{subsec:faculty}
Support from faculty could greatly advance this study and what RIT SPEX is capable robotics wise. It is not necessary but would be very helpful. Professors could help with funding, rover design, hardware selection and general advice. Rovers and robotics are new to SPEX and someone with experience could help this. The team should reach out to professors for advice
and help. There are many professors at RIT with interest in robotics or computer vision. With some recommendations from SPEX alumni we have a short list of professors who might be interested.

\begin{table}[ht!]
    \caption{Potential Faculty Support}
    \centering
    {\renewcommand{\arraystretch}{1.5}
    \begin{tabularx}{\linewidth}{LLL} 
    \hline
    \textbf{Professor} & \textbf{Department / Field} & \textbf{Email Address} \\
    \hline
    Dr. Ugur Sahin & Mathematics (Researches Computer Vision) & us-grd@cs.rit.edu \\
    Dr Ferat Sahin & KGCOE (Robotics) & feseee@rit.edu \\
    \hline
    \end{tabularx}
    }
\label{tab:fac-sup}
\end{table}

\subsection{Long-Term Vision}
\label{subsec:vision}
The long-term vision of this project is to open RIT Space Exploration up to a new area of projects and development. Robotics and rovers are at the core of deep space exploration
and most science missions. The University Rover Challenge is hosted by the Mars Society. It
features very advanced and expensive rovers. The ability to compete in this top-tier level of university robotics allows RIT SPEX to develop the next generation of leaders and engineers for the field of robotics. These rovers feature autonomous navigation, fine control robotics, scientific drill and more. It is a worthy long-term goal.


\section*{Acknowledgements}
The author would like to thank SPEX alumni Phil Linden for creating the PDD template, Anthony Hennig for founding RIT Space Exploration, and all the SPEX members that continue to invest their time and energy into the pursuit of Space Exploration.
\section{References}
\begin{itemize}
  \item \href{http://urc.marssociety.org/files/University%20Rover%20Challenge%20Rules%202018.pdf}{University Rover Challenge Download the Requirements and Guidelines PDF}
  \item \href{http://urc.marssociety.org/home/q-a}{RC2018 Q\&A}
\end{itemize}
\onecolumn
\appendices{}

\end{document}
