%% *************************************************************************
%%
%% This is the PDD for the Rovers Study - Fall 2018
%%
%% This document uses IEEEtran.cls, the official IEEE LaTeX class
%% for authors of the Institute of Electrical and Electronics Engineers
%% (IEEE) Transactions journals and conferences.
%%
%% *************************************************************************

%% *************************************************************************
% LaTeX REFERENCES
% ----------------
%   Intro to LaTeX: http://www.rpi.edu/dept/arc/docs/latex/latex-intro.pdf
%   Comprehensive LaTeX symbol list: http://tug.ctan.org/info/symbols/comprehensive/symbols-a4.pdf
%% *************************************************************************

% tell \LaTeX what kind of formatting to use
\documentclass[conference]{IEEEtran} % http://www.ctan.org/pkg/ieeetran
\usepackage{blindtext} % enable placeholder text generator
\usepackage{graphicx} % enable toolbox for embedding figures and pictures
\usepackage{nomencl} % enable package for adding a list of variables and constants at the beginning, aka "nomenclature"
\usepackage{siunitx} % enable package for easily formatting units
\usepackage{hyperref} % enable package for cross-referencing figures, sections, references etc.
% how to use hyperref: http://www2.washjeff.edu/users/rhigginbottom/latex/resources/lecture09.pdf
\usepackage[T1]{fontenc} % change text encoding to make it more crisp
\usepackage{etoolbox} % enable conditionals for help text
\usepackage{booktabs} % make beautiful tables!

% initialize nomenclature package
\makenomenclature{}

% set title. choose something as descriptive and precise as possible. Descriptive > sounding cool. remember this!
\title{Rover}


\author{
  % List the authors of the design document. 
  \IEEEauthorblockN{% This block is for author Names.
    Thomas~Hall\IEEEauthorrefmark{1}
  }
  \IEEEauthorblockA{% This block is for the author Affiliations, aka department and university
    RIT Space Exploration, Rochester Institute of Technology \\ %\\ starts a new line
    Rochester, N.Y. \\
    Email:
    \IEEEauthorrefmark{1}tjh2822@g.rit.edu
  }
}

% page header for pages other than cover page
\markboth{Project Design Document Standard}%
{Hall \MakeLowercase{\textit{et al.}}: RIT Space Exploration}

% Initial setup is over, start building the document itself
\begin{document}
\maketitle%
% correct bad hyphenation here, separated by spaces
\hyphenation{explor-ation}

\begin{abstract}
  Begin the design and construction of a mock rover. A rover would be an area of space exploration completely new to SPEX, because of this there are a lot of unknowns that need to be answered before starting this project. Because of this the project will require a large amount of prototyping and testing. The team would look at the talent and skills of RIT Space Exploration and answer the question: what caliber of rover are we able to make. It would employ a multidisciplinary team to cover for construction and development. The project would be slated to run for two semesters. The rover would be extremely promotable and would look great at Imagine RIT. 
      % The abstract is a brief summary of the design document. Typically it includes the purpose of the design document, key goals or objectives, and justifications.
\end{abstract}

\label{sec:nomenclature}
\newcommand{\nomunit}[1]{%
\renewcommand{\nomentryend}{\hspace*{\fill}#1}}
\renewcommand{\nompreamble}{}

\nomenclature{RIT}{Rochester Institute of Technology}
\nomenclature{SPEX}{RIT Space Exploration}
\nomenclature{PDD}{Project Design Document}
% Below are examples of using nomenclature for math symbols and constants or units
\nomenclature{$\dot{m}$}{Mass flow rate
  \nomunit{\,\si{\kilo\gram\per\second}}}
\nomenclature{$c$}{Speed of light
 \nomunit{\,\SI{2.9979e8}{\meter\per\second}}}
\printnomenclature{}

% The sections included here are required. Additional sections and subsections may be added as necessary.
\section{Introduction}
\label{sec:introduction}
  % The introduction is a place to give background and context before diving into the subject matter.
  % Establish context for the work you are about to propose and the main ideas of the proposition itself.

\IEEEPARstart{R}{obotics} and by extension rovers are a tremendously important part of space exploration. 
This is also an area that RIT Space Exploration has very little experience with project-wise. 
The purpose of this project is to assess and assert the capability of RIT SPEX in regard to the construction and fabrication of a mock-rover and then begin construction. 
As this is a new area to RIT SPEX, prototyping and experimenting is necessary to most effectively accomplish this goal. 
This project will look at the unknowns, technical challenges, project management, and member skills of RIT Space Exploration in regards to a University Rover Competition (URC) capable rover.
The project is intended to last 2 semesters with the deliverable of a basic rover and a technical report. 
The project is also intended to lead into a future rover project with expanded capabilities.

\section{Primary Objective}
\label{sec:primary-obj}
  % At the end of the day, whether the project ``succeeds'' or ``fails'' is judged against the objectives it sought to meet.
  % Note that results that contradict expectations/hypotheses are not failures if the scientific \& engineering methods are followed along the way.
  % Sometimes our expectations are wrong and that can be just as successful as getting data we thought we'd see.
  % What matters are what questions you intend to answer.
  % This is the main purpose or main goal the project hopes to achieve.

The goal of this project is deliver a functioning mock rover. The rover will allow the team to develop skills and knowledge to build a more feature completer rover in the future. The details are defined in the minimum viable product section. 

%\begin{figure}
%  \includegraphics[width=\linewidth]{figs/project-life-cycle.png}
%  \caption{A PDD is the first piece of documentation to be archived in the project life cycle. Since the life cycle can be iterative, a new design document may also refer to one or more previous SPPs.}
%\label{fig:lifecycle}
%\end{figure}

\section{Secondary Objectives}
\label{sec:secondary-obj}
  % Secondary Objectives are lower priority or bonus objectives that are significant but not the main focus of the project. This template does not have secondary objectives.
The study will also investigate the University Rover Competition (URC) as hosted by the mars Society as a potential long term goal. The competition is held annually and features 4 very intense competitions. Such a rover would need to be fully self navigating, perform scientific analysis on soil samples, and having a robotic arm capable of fine motor control. Such a project would be among the most ambitious SPEX has ever attempted. The skills learned from this project should be applicable to such a future project. 

\section{Minimum Viable Product}
\label{sec:mvp}
  % Mvp
  The minimum viable product (MVP) is defined by the following features: 

\begin{itemize}
  \item Rocker-bogie-like suspension
  \item Ability to control remotely (such as a gamepad)
  \item Ability to traverse terrain such as gravel, cement, and asphalt
  \item Technical report and rover git as deliverable
  \item Physical rover demo present at Imagine RIT
\end{itemize}

  The MVP is intended to be flexible in implementation depending on funding. The size and materials are not defined because of this. Should the funding allow for it there are many improvements that can be made beyond the MVP and quality materials. 

\begin{itemize}
  \item Lidar based CV for autonomous or partially autonomous navigation
  \item Improved navigation with use of GPS system
  \item Robotic arm with 3+ degrees of freedom
    \item Drill or grasper attachment
  \item Soil sample analysis with archemetes screw
  \item Larger / faster rover
  \item Solar cells for additional long-term navigation
  \item better control interface  
\end{itemize}

\section{Benefit to SPEX}
\label{sec:benefit}
% Explains the benefit to RIT SPEX

Rovers are a huge part of space exploration. It is also an area that SPEX is not currently involved with. 
It would be beneficial to our members to get some experience in this area.
A rover would look super good for SPEX at Imagine RIT. 
The rover would be rather large and would attract many eyes. We could even demo it outside if there is sufficient space. 
A rover would be very easy to get video and photos of for SPEX promotions. 
Having a rover is also another opportunity for SPEX to fundraise. 
There is plenty of space to place company logos on the body of the rover. 
It would also allow for SPEX to reach out to robotics companies.
The self driving component would be the heaviest computer science project SPEX would have attempted, this would help with retention of CS and SE majors. 
Machine learning and artificial intelligence are at the forefront of computer science right now. 
They are heavily desired in industry including space exploration.

% Subsections

\subsection{Mindset}
\label{subsec:mindset}
The purpose of this project is to investigate and implement a mock rover.
Because of this the team members must be in the mindset to analyze each part of the project and identify as many problem areas that need answers as possible. 
This means being specific on how we are going to accomplish our goals. 
What material, what algorithm, with what method will we be accomplishing this.
It is very easy to over estimate the team's ability especially when the project is so new to the team.

\subsection{Traceability}
\label{subsec:traceability}

It is important that the team members document the sources they use to gather information. 
There will be a Google Drive folder that will hold notes with links to any books, articles, media or their sources that are relevant to the rover project. 
The software and hardware will be tracked on GitHub. 
This includes the technical report.

\subsection{Assessing Technical Skills}
\label{subsec:assessing-technical-skills}

Building a rover, especially one built to URC specifications requires a large team with a diverse set of technical skills. 
This project will require SPEX to look at and asses what areas are sufficient and what areas need to be developed in order to build a rover. 
It will also let us know to what spec we are currently capable of building such a rover. 

\subsection{Accessibility}
\label{subsec:plug-n-play}

The project would be very ambitious. It would require at a minimum a handful (4-5) members each with different areas of expertise. 
The members would need that knowledge as well as \LaTeX{}, notetaking, and good research practices. 
This would be to deliver the MVP.
A larger team would be able to accomplish more.  

\section{Implementation}
\label{sec:implementation}
  % What path do you anticipate the project to take?

This project would involve divvying out the different project areas to the different team members or small groups and each week meeting to discuss progress. 
The team would take this information, refine it and put it into the technical report along with construction.  

\subsection{Deliverables}
\label{subsec:deliverables}
  % When all is said and done, what will you have to show for it?
  % Examples: Hardware, software, poster, ImagineRIT demo, presentations, technical papers...
The primary deliverables of this project will be the technical report and the rover itself by the end of the second semester. 
It is also noted that the materials that the team will come across or notes should also be saved in an Google Drive folder for future reference.  
The team will also present at SPEX design reviews and the weekly checkups at general meetings with the areas the team members are working on that week.

\subsection{Milestones}
\label{subsec:milestones}
  % Be as detailed as you can, but it's okay if there are unknowns.
  % At the very least, specify how many semester you expect the project to take until it reaches completion.
The largest milestones for the technical report would be the would be draft and completion. 
The rover itself would include the various CAD models of the rover, software milestones, electrical system completion, and drive train system. 

\section{Externalities}
  % Things not directly related to the work or outcomes, but related to the project as a whole.
\subsection{Prerequisite Skills}
  % Which skills do team members need to have before work can start (not including skills that will be learned ``on the job'')?
The project is going to require members that are knowledgeable in mechanical design and manufacturing, software engineering, machine learning, electrical engineering, and project management. These members will also need to be experienced in research and design document work. Though this seems simple, finding SPEX members to do this well may be a challenge duye to competition from other projects. 

\subsection{Funding Requirements}
  % Estimate costs that would be needed to meet objectives.

The rover project is looking at a minimal funding amount of \$500.00 USD from RIT SPEX. In order to give a better deliverable the team will also be reaching out to a large list of companies looking for further funding through sponsorship. 

\subsection{Faculty Support}
  % Identify faculty that will be involved (or would need to be involved) to meet objectives.
  % Note that if a professor is the Principal Investigator (P.I.) for a project, there still needs to be a student as the SPEX Project Champion.
Support from faculty could greatly advance this study and what RIT SPEX is capable robotics wise. The team should reach out to professors for advice and help.

\subsection{Long-Term Vision}
\label{sec:vision}
The long-term vision of this project is to open RIT Space Exploration up to a new area of projects and development. 
Robotics and rovers are at the core of deep space exploration and most science missions. 
The University Rover challenge is also a worthy long term goal. 

\section*{Acknowledgements}
The author would like to thank SPEX alumni Phil Linden for creating the PDD template, Anthony Hennig for founding RIT Space Exploration, and all the SPEX members that continue to invest their time and energy into the pursuit of space exploration.

\bibliographystyle{IEEEtran}
\bibliography{sample-with-examples}

\onecolumn
\appendices{}
%\section{Project Life Cycle}
%\begin{figure}[h]
%  \centering
%  \includegraphics[]{figs/project-life-cycle.png}
%  \caption{Enlarged version of the diagram in \autoref{fig:lifecycle}.}
%\end{figure}

\end{document}
