%% *************************************************************************
%%
%% This is the PDD for the Rovers Study - Fall 2018
%%
%% This document uses IEEEtran.cls, the official IEEE LaTeX class
%% for authors of the Institute of Electrical and Electronics Engineers
%% (IEEE) Transactions journals and conferences.
%%
%% *************************************************************************

%% *************************************************************************
% LaTeX REFERENCES
% ----------------
%   Intro to LaTeX: http://www.rpi.edu/dept/arc/docs/latex/latex-intro.pdf
%   Comprehensive LaTeX symbol list: http://tug.ctan.org/info/symbols/comprehensive/symbols-a4.pdf
%% *************************************************************************

% tell \LaTeX what kind of formatting to use
\documentclass[conference]{IEEEtran} % http://www.ctan.org/pkg/ieeetran
\usepackage{tabularx} % fix tables
  \newcolumntype{L}{>{\raggedright\arraybackslash}X}
\usepackage{blindtext} % enable placeholder text generator
\usepackage{graphicx} % enable toolbox for embedding figures and pictures
\usepackage{nomencl} % enable package for adding a list of variables and constants at the beginning, aka "nomenclature"
\usepackage{siunitx} % enable package for easily formatting units
\usepackage{hyperref} % enable package for cross-referencing figures, sections, references etc.
% how to use hyperref: http://www2.washjeff.edu/users/rhigginbottom/latex/resources/lecture09.pdf
\usepackage[T1]{fontenc} % change text encoding to make it more crisp
\usepackage{etoolbox} % enable conditionals for help text
\usepackage{booktabs} % make beautiful tables!

% initialize nomenclature package
\makenomenclature{}

% set title. choose something as descriptive and precise as possible. Descriptive > sounding cool. remember this!
\title{Rovers}


\author{
  % List the authors of the design document. 
  \IEEEauthorblockN{% This block is for author Names.
    Thomas~Hall\IEEEauthorrefmark{1}
  }
  \IEEEauthorblockA{% This block is for the author Affiliations, aka department and university
    RIT Space Exploration, Rochester Institute of Technology \\ %\\ starts a new line
    Rochester, N.Y. \\
    Email:
    \IEEEauthorrefmark{1}tjh2822@g.rit.edu
  }
}

% page header for pages other than cover page
\markboth{Project Design Document Standard}%
{Hall \MakeLowercase{\textit{et al.}}: RIT Space Exploration}

% Initial setup is over, start building the document itself
\begin{document}
\maketitle
% correct bad hyphenation here, separated by spaces
\hyphenation{explor-ation}

\begin{abstract}
  Begin the design and construction of a mock rover. A rover would be an area of space exploration completely new to SPEX, because of this there are a lot of unknowns that need to be answered before starting this project. Because of this the project will require a large amount of prototyping and testing. The team would look at the talent and skills of RIT Space Exploration and answer the question: what caliber of rover are we able to make. It would employ a multidisciplinary team for construction and development. The project would be slated to run for two semesters. The rover would be extremely promotable and would look great at Imagine RIT. 
      % The abstract is a brief summary of the design document. Typically it includes the purpose of the design document, key goals or objectives, and justifications.
\end{abstract}

\label{sec:nomenclature}
\newcommand{\nomunit}[1]{%
\renewcommand{\nomentryend}{\hspace*{\fill}#1}}
\renewcommand{\nompreamble}{}

\nomenclature{RIT}{Rochester Institute of Technology}
\nomenclature{SPEX}{RIT Space Exploration}
\nomenclature{PDD}{Project Design Document}
% Below are examples of using nomenclature for math symbols and constants or units
\nomenclature{$\dot{m}$}{Mass flow rate
  \nomunit{\,\si{\kilo\gram\per\second}}}
\nomenclature{$c$}{Speed of light
 \nomunit{\,\SI{2.9979e8}{\meter\per\second}}}
\printnomenclature{}

% The sections included here are required. Additional sections and subsections may be added as necessary.
\section{Introduction}
\label{sec:introduction}
  % The introduction is a place to give background and context before diving into the subject matter.
  % Establish context for the work you are about to propose and the main ideas of the proposition itself.

\IEEEPARstart{R}{obotics}  and by extension rovers are a tremendously important
part of space exploration. 
This is also an area that RIT
Space Exploration has very little experience with project-wise.
The purpose of this project is to assess and assert the capability
of RIT SPEX regarding the construction and fabrication of a mock-rover and then begin construction. 
As this is a new area to RIT SPEX, prototyping and experimenting is necessary to most effectively accomplish this goal. 
This project will look at the unknowns, technical challenges, project management, and member skills of RIT Space Exploration in regard to a University Rover Competition (URC) capable rover. 
The URC is hosted by The Mars Society annually. 
It is very competitive and requires a very complex and expensive rover that is currently out of scope for SPEX. 
This rover would need to be capable of full autonomous navigation, soil collection and analysis, and precise robotics. 
This project is the first step in building the skills for such a team.


\section{Primary Objective}
\label{sec:primary-obj}
  % At the end of the day, whether the project ``succeeds'' or ``fails'' is judged against the objectives it sought to meet.
  % Note that results that contradict expectations/hypotheses are not failures if the scientific \& engineering methods are followed along the way.
  % Sometimes our expectations are wrong and that can be just as successful as getting data we thought we'd see.
  % What matters are what questions you intend to answer.
  % This is the main purpose or main goal the project hopes to achieve.

The goal of this project is delivering a functioning mock rover.
We define mock rover as a functioning rover built to specifications
for a mock mission as it will not be launched. The rover
will allow the team to develop skills and knowledge to build a
more feature completer rover in the future. These skills include
rover fabrication and design, autonomous navigation, electrical
system design, rover testing, computer vision, and robotics.
The rover will be testing using events designed like
the URC but scaled down to fit the scope of this rover. We
will be using time as a metric as well as binary metrics for
completion of the parts of the competition. This includes ’does
the rocker-bogie system works as intended’, ’will the rover stop
when directed at an obstacle’, etc. The details are defined in
the minimum viable product section.


\subsubsection{Minimum Viable Product}
  % Mvp
The minimum viable product (MVP) has been defined by the basics of what can be called a 'functioning rover'. This includes features like suspension, motors, power and controls. The MVP is defined by the following features: 

\begin{table}[hb!]
    % the "h" in these brackets tells LaTeX to put the table Here. Try [t] for top and [b] for bottom,
    % or [hbp] for "here, or if you can't do that put it at the bottom of the page, or if you can't do that put it on its own page.
    % Here we've also used an "!" to yell at LaTeX to DO THIS OR ELSE!
    \caption{Minimum Viable Product}
    \centering
    %%\begin{tabular}{l\p{15mm}}%%{\textwidth}%%{@{}XX@{}}
    % the letters here ^^^^ designate the columns.
    % (l=left align, c=center, r=right align)
    % the weird @{} thingies tell LaTeX to not have left-right padding between cells
    % so cells butt up right against the edge
    {\renewcommand{\arraystretch}{1.2}
    \begin{tabularx}{\linewidth}{LL} 
    \hline
    \textbf{Feature} & \textbf{Reason} \\
    \hline
    Rocker-bogie-like suspension & This is the NASA perfered rover suspension style. \\
    \hline
    Ability to control remotely (such as a gamepad) & Remote controle is essential for rovers. \\
    \hline
    Ability to traverse terrain such as gravel, cement, and asphalt & Multi-terrain traversal is important for rovers. \\
    \hline
    Technical report on GitHub & This will allow for future teams to easily learn from htis teams experience. \\
    \hline
    Rover hardware  & Having a physical rover is important for a rover project as it allows for lessons to be learned in construction. \\ 
    \hline
    Present at Imagine RIT & This is essential for promotional value. \\
    \hline
    \multicolumn{2}{c}{\textbf{Improvements:}} \\
    \hline
    Lidar based CV & Autonomous or partially autonomous navigation for future URC readiness. \\
    \hline
    Improved navigation with use of GPS system & Additional navigation options. \\
    \hline
    Robotic arm with 3+ degrees of freedom & URC readiness and rover capabilities. \\ 
    \hline
    Drill or grasper attachment & Increased capabilities of the robotic arm. \\
    \hline
    Soil sample analysis with Archimedes' screw & Scientific possibilities. \\
    \hline
    Larger and or faster rover & Able to traverse obstacles of greater magnitude or hold more scientific equipment for future expansion. \\
    \hline
    Solar cells & Additional long-term navigation viability and experience. \\
    \hline
    Web control interface with GUI & Easier navigation. \\
    \hline
    Durable construction & Carbon fiber or similar materials for long term viability. \\
    \hline
    On board batteries & Lithium Ion, 18650s or similar battery technology for remote abilities. \\
    \hline
    \end{tabularx}
    }
\label{tab:mvp-one}
\end{table}

\subsubsection{Improvements}

  The MVP is intended to be flexible in implementation depending on funding. The size and materials are not defined because of this. Should the funding allow for it there are many improvements that can be made beyond the MVP. 

%\begin{figure}
%  \includegraphics[width=\linewidth]{figs/project-life-cycle.png}
%  \caption{A PDD is the first piece of documentation to be archived in the project life cycle. Since the life cycle can be iterative, a new design document may also refer to one or more previous SPPs.}
%\label{fig:lifecycle}
%\end{figure}

\section{Secondary Objectives}
\label{sec:secondary-obj}
  % Secondary Objectives are lower priority or bonus objectives that are significant but not the main focus of the project. This template does not have secondary objectives.
The study will also investigate the University Rover Competition (URC) as hosted by the Mars Society as a potential long term goal. 
The team will accomplish this by evaluating how many of the MVP improvements are accomplished. 
If none of these improvements are not able to be implemented then the team is not near URC ready.
The competition is held annually and features four very intense competitions. 
Such a rover would need to be fully self navigating, perform scientific analysis on soil samples, and having a robotic arm capable of fine motor control. 
Such a project would be among the most ambitious SPEX has ever attempted. The skills learned from this project should be applicable to such a future project. 
This secondary objective is directly related to the long-term vision of the rover project. 

\section{Benefit to SPEX}
\label{sec:benefit}
% Explains the benefit to RIT SPEX

Rovers are a huge part of space exploration. 
It is also an area that SPEX is not currently involved with. 
It would be beneficial to our members to get some experience in this area.
\subsubsection{Promotions and Funding}
A rover would look super good for SPEX at Imagine RIT. 
The rover would be rather large and would attract many eyes. We could even demo it outside if there is sufficient space. 
A rover would be very easy to get video and photos of for SPEX promotions. 
Having a rover is also another opportunity for SPEX to fundraise. 
There is plenty of space to place company logos on the body of the rover. 
It would also allow for SPEX to reach out to robotics companies.
\subsubsection{Computer Science}
The potential self-driving component would be the heaviest computer science project SPEX would have attempted, this would help with retention of CS and SE majors. 
Some of the potential self-driving features could be implemented with artificial intelligence.
Machine learning and artificial intelligence are at the forefront of computer science right now. 
They are heavily desired in industry including space exploration.
\subsubsection{Robotics}
\label{robotics}
Robotics are also a heavily desired area in the space industry. Currently RIT SPEX does not have any projects with robotics as a focus. Robotics is also a potential area for collaboration with faculty as there are RIT faculty that are investigating this area. 

% Subsections

\subsection{Mindset}
\label{subsec:mindset}
The purpose of this project is to investigate and implement a mock rover.
Because of this the team members must be in the mindset to analyze each part of the project and identify as many problem areas that need answers as possible. 
This means being specific on how we are going to accomplish our goals.
We want our team asking the question:"what material, what algorithm, with what method will we be accomplishing this?"
It is very easy to over estimate the team's ability especially when the project is so new to the team.

\subsection{Traceability}
\label{subsec:traceability}

It is important that the team members document the sources they use to gather information. 
There will be a Google Drive folder that will hold notes with links to any books, articles, media or their sources that are relevant to the rover project. 
The software and hardware will be tracked on GitHub. 
This includes the technical report.

\subsection{Accessibility}
\label{subsec:plug-n-play}

The project would be very ambitious. It would require at a minimum a handful (4-5) members each with different areas of expertise. 
The members would need that knowledge as well as \LaTeX{}, notetaking, and good research practices. 
This would be to deliver the MVP.
A larger team would be able to accomplish more.  

\section{Implementation}
\label{sec:implementation}
  % What path do you anticipate the project to take?

This project would involve divvying out the different project areas to the different team members or small groups and each week meeting to discuss progress. 
The team would take this information, refine it and put it into the technical report along with construction.  

\subsection{Deliverables}
\label{subsec:deliverables}
  % When all is said and done, what will you have to show for it?
  % Examples: Hardware, software, poster, ImagineRIT demo, presentations, technical papers...

The deliverable includes the technical report, Rover software, and Rover hardware.
The primary deliverables of this project will be the technical report and the rover itself by the end of the second semester.
It is also noted that the materials that the team will come across or notes should also be saved in an Google Drive folder for future reference. 
The team will also present at SPEX design reviews and the weekly checkups at general meetings with the areas the team members are working on that week.
The project is intended to last 2 semesters with the deliverable of a basic rover and a technical report.   

\subsubsection{Technical Report}
\label{techreport}

The technical report of the rover is a design document complete with the hardware, software, and electrical components of the rover. This report is important as it allows for a future team to easily learn from the past team's experience. This Technical Report will be hosted on GitHub. This will also include the results of the rover tests and the teams recommendations  based on them. 

\subsubsection{Rover Software}
\label{roversoftware}
The rover software has to be written and hosted on GitHub. This will include rover features such as navigation, control, and power regulation. 

\subsubsection{Rover Hardware}
\label{roverhardware}
The rover will need to be constructed and tested. Because of this a physically constructed rover is needed. This will be necessary for ImagineRIT for cool factor. 

\subsection{Milestones}
\label{subsec:milestones}
  % Be as detailed as you can, but it's okay if there are unknowns.
  % At the very least, specify how many semester you expect the project to take until it reaches completion.
The largest milestones for the technical report would be the would be draft and completion. 
The rover itself would include the various CAD models of the rover, software milestones, electrical system completion, and drive train system. 

\section{Externalities}
  % Things not directly related to the work or outcomes, but related to the project as a whole.
\subsection{Prerequisite Skills}
  % Which skills do team members need to have before work can start (not including skills that will be learned ``on the job'')?
The project is going to require members that are knowledgeable in mechanical design and manufacturing, software engineering, (potentially) machine learning, electrical engineering, and project management. These members will also need to be experienced in research and design document work. Though this seems simple, finding SPEX members to do this well may be a challenge due to competition from other projects. 

Building a rover, especially one built to URC specifications requires a large team with a diverse set of technical skills. 
This project will require SPEX to look at and asses what areas are sufficient and what areas need to be developed in order to build a rover. 
It will also let us know to what spec we are currently capable of building such a rover. 

\subsection{Funding Requirements}
  % Estimate costs that would be needed to meet objectives.

The rover project is looking at a minimal funding amount of \$500.00 USD from RIT SPEX. This funding is needed to make a rover that fulfills the MVP defined in the Primary Objective section. In order to give a better deliverable the team will also be reaching out to a large list of companies looking for further funding through sponsorship. 

\subsection{Faculty Support}
  % Identify faculty that will be involved (or would need to be involved) to meet objectives.
  % Note that if a professor is the Principal Investigator (P.I.) for a project, there still needs to be a student as the SPEX Project Champion.
Support from faculty could greatly advance this study and what RIT SPEX is capable robotics wise. It is not necessary but would be very helpful. Professors could help with funding, rover design, hardware selection and general advice. Rovers and robotics are new to SPEX and someone with experience could help this. The team should reach out to professors for advice and help. There are many professors at RIT with interest in robotics or computer vision. With some recemendations from SPEX alumni we have a short list of professors who might be interested. 

%%\begin{itemize}
%%  \item Dr. Ugur Sahin | Mathematics (Researches Computer Vision) | us-grd@cs.rit.edu
%%  \item Dr Ferat Sahin | KGCOE (Robotics) | feseee@rit.edu
%%\end{itemize}

\begin{table}[hb!]
    % the "h" in these brackets tells LaTeX to put the table Here. Try [t] for top and [b] for bottom,
    % or [hbp] for "here, or if you can't do that put it at the bottom of the page, or if you can't do that put it on its own page.
    % Here we've also used an "!" to yell at LaTeX to DO THIS OR ELSE!
    \caption{Potential Faculty support.}
    \centering
    %%\begin{tabular}%%{\textwidth}{@{}XX@{}}
    % the letters here ^^^^ designate the columns.
    % (l=left align, c=center, r=right align)
    % the weird @{} thingies tell LaTeX to not have left-right padding between cells
    % so cells butt up right against the edge
    {\renewcommand{\arraystretch}{1.2}
    \begin{tabularx}{\linewidth}{LLL} 
    \hline
    \textbf{Professor} & \textbf{Department / Feild} & \textbf{Email Address} \\
    \hline
    Dr. Ugur Sahin & Mathematics (Researches Computer Vision) & us-grd@cs.rit.edu \\
    \hline
    Dr Ferat Sahin & KGCOE (Robotics) & feseee@rit.edu \\
    \hline
    \end{tabularx}
    }
\label{tab:fac-sup}
\end{table}

\subsection{Long-Term Vision}
\label{sec:vision}
The long-term vision of this project is to open RIT Space Exploration up to a new area of projects and development. 
Robotics and rovers are at the core of deep space exploration and most science missions. 
The project is also intended to lead into a future rover project with expanded capabilities.
The University Rover Challenge is hosted by the mars society. 
It features very competitive and expensive rovers. 
These rovers feature autonomous navigation, fine control robotics, scientific drill and more. 
It is a worthy long term goal. 


\section*{Acknowledgements}
The author would like to thank SPEX alumni Phil Linden for creating the PDD template, Anthony Hennig for founding RIT Space Exploration, and all the SPEX members that continue to invest their time and energy into the pursuit of space exploration.

\onecolumn
\appendices{}
%\section{Project Life Cycle}
%\begin{figure}[h]
%  \centering
%  \includegraphics[]{figs/project-life-cycle.png}
%  \caption{Enlarged version of the diagram in \autoref{fig:lifecycle}.}
%\end{figure}

\end{document}
