%% *************************************************************************
%%
%% This is a derivative work of the RIT Space Exploration Standard defining 
%% guidelines for content and formatting of project design documents.
%%
%% This document uses IEEEtran.cls, the official IEEE LaTeX class
%% for authors of the Institute of Electrical and Electronics Engineers
%% (IEEE) Transactions journals and conferences.
%%
%% *************************************************************************

%% *************************************************************************
% LaTeX REFERENCES
% ----------------
%   Intro to LaTeX: http://www.rpi.edu/dept/arc/docs/latex/latex-intro.pdf
%   Comprehensive LaTeX symbol list: http://tug.ctan.org/info/symbols/comprehensive/symbols-a4.pdf
%% *************************************************************************

% tell \LaTeX what kind of formatting to use
\documentclass[conference]{IEEEtran} % http://www.ctan.org/pkg/ieeetran
\usepackage{blindtext} % enable placeholder text generator
\usepackage{graphicx} % enable toolbox for embedding figures and pictures
\usepackage{nomencl} % enable package for adding a list of variables and constants at the beginning, aka "nomenclature"
\usepackage{siunitx} % enable package for easily formatting units
\usepackage{hyperref} % enable package for cross-referencing figures, sections, references etc.
% how to use hyperref: http://www2.washjeff.edu/users/rhigginbottom/latex/resources/lecture09.pdf
\usepackage[T1]{fontenc} % change text encoding to make it more crisp
\usepackage{etoolbox} % enable conditionals for help text
\usepackage{booktabs} % make beautiful tables!
\usepackage{tabularx}

% initialize nomenclature package
\makenomenclature{}

% set title. choose something as descriptive and precise as possible. Descriptive > sounding cool. remember this!
\title{RIT Observatory Projects Fall Semester 2018}


\author{
  % List the authors of the design document. The Champion should go first.
  % The \$~\$ markers tell \LaTeX{} to treat the text inside to be treated as a math expression. This way you can use operators like \textcaret{} to place characters as superscripts.
  % Some \LaTeX{} templates handle the author block in different ways. For example, the \href{http://www.worldscientific.com/worldscinet/jai}{Journal of Astronomical Instrumentation} requires the authors' addresses and emails to be included as well.
  % The \textbackslash{}thanks command puts the contents inside those brackets in a footnote at the bottom of the first page. Technically speaking, \textbackslash{}thanks is just a specially formatted footnote.
  % IEEE also has a ``long form'' author block for many authors. Check here for more information:
  % \url{https://tex.stackexchange.com/questions/156523/multiple-authors-with-common-affiliations-in-ieeetran-conference-template}
  % Read here for a more advanced options to modifying footnotes in the author block:  \url{http://tex.stackexchange.com/questions/826/symbols-instead-of-numbers-as-footnote-markers}
  %   Here, we use the IEEE long-form author block.
  \IEEEauthorblockN{% This block is for author Names.  %the number in the bracket is a reference number to identify this footnote. \LaTeX will figure out what symbol to put there.
    Amber~Dubill\IEEEauthorrefmark{1},
    Dr. Jennifer Connelly\IEEEauthorrefmark{2},
    and
    Dr. Micheal Richmond\IEEEauthorrefmark{3},
  }
  \IEEEauthorblockA{% This block is for the author Affiliations, aka department and university
    RIT Space Exploration, Rochester Institute of Technology \\ %\\ starts a new line
    Rochester, N.Y. \\
    Email:
    \IEEEauthorrefmark{1}ald4035@rit.edu,
    \IEEEauthorrefmark{2}jlcsps@rit.edu,
    \IEEEauthorrefmark{3}mwrsps@rit.edu,
  }
  %%   Below, we use the short-form author block and basically hack it to suit our needs.
  % Philip~Linden$^{*\dagger}$%
  %   \thanks{$^{*}$Project Champion}%
  %   \thanks{$^{\dagger}$BS/MEng '17, Mechanical Engineering},
  % Austin~Bodzas$^{\ddagger}$%
  %   \thanks{$^{\ddagger}$BS '19, Computer Science},
  % Drew~Walters$^{\S}$%
  %   \thanks{$^{\S}$BS '18, Mechanical Engineering Technology},
  % T.J.~Tarazevits$^{**}$%
  %   \thanks{$^{**}$BS '19, Game Design \& Development}%

  %%   If there are many authors, consider using symbolic, numeric (aka arabic),  alphabet footnotes or a combination thereof.
  %% the recommended order for symbolic footnotes is
  %%   (1) asterisk        *   *
  %%   (2) dagger          †   \dagger
  %%   (3) double dagger   ‡   \ddagger
  %%   (4) section symbol  §   \S
  %%   et cetera. For higher counts, use 2x symbols (1)-(4) (i.e. (5) two asterisks **). Keep cycling through (1)-(4) using 3x, 4x, and so on.
  %%   Note that these symbol codes work in math mode and text mode.
  %%   There are ways to make LaTeX do this for you, but it is more advanced and not entirely necessary, especially for short author lists. Not worth the hassle, in my opinion.
}
% page header for pages other than cover page
\markboth{Project Design Document Standard}%
{Linden \MakeLowercase{\textit{et al.}}: RIT Space Exploration}

% Initial setup is over, start building the document itself
\begin{document}
\maketitle%
% correct bad hyphenation here, separated by spaces
\hyphenation{explor-ation}

\begin{abstract}
  The RIT observatory currently does its best to host numerous students with the limited faculty and resources they possess.
  There is a plethora of smaller projects that would greatly improve the ability of the observatory while providing engineering experience to students in the realm of control systems, opto-mechanics, and instruments.
  This would also allow students to become familiar with the various equipment the observatory has to offer.
  The over arching wish of the faculty at the observatory is to complete a fully automated dome control, so that remote observing would become feasible.
  This task has been started, but is still missing a robust feedback system, some mechanical interfaces, and software interfacing with the telescope protocols.
  Select projects have been proposed for completion in the 2018 fall semester.

      % The abstract is a brief summary of the design document. Typically it includes the purpose of the design document, key goals or objectives, and justifications.
      % Be sure not to confuse the abstract with the introduction.
      % It is easiest to write the abstract after the rest of the paper has been written.
      % That way you can choose key information from the sections that you've already completed and string them together in the abstract.
      % Consider the abstract to be your elevator pitch to anyone reading this design document.
      % What are they reading?
      % What is the goal?
      % Why is it worth my time?
      % The abstract is what will show up in Google results and other search engines, and what people will read when they are deciding what is worth their time and brain power.
\end{abstract}

\label{sec:nomenclature}
\newcommand{\nomunit}[1]{%
\renewcommand{\nomentryend}{\hspace*{\fill}#1}}
\renewcommand{\nompreamble}{
    % If you include mathematical expressions or express variables in the design document, list them with their corresponding definitions here as a list.
    % The two lines below make it look nice when defining units/values to constants.

    % Note that math terms and non-math terms are separated and alphabetized, regardless of the order in which they are defined. (Recall terms \$like this\$ are in the math environment)
    % Read more about advanced nomenclature formatting here:\\
    % \url{https://www.sharelatex.com/learn/Nomenclatures}
  }
\nomenclature{RIT}{Rochester Institute of Technology}
\nomenclature{SPEX}{RIT Space Exploration}
\nomenclature{PDD}{Project Design Document}
\nomenclature{CCD}{Charged Coupled Device}
\nomenclature{MSD}{Multidisciplinary Senior Design}
\nomenclature{SG}{Student Government}
\printnomenclature{}


% HELPFUL HINT: If you get the warning ``Command terminated with space.'' when using a \command try placing ``%'' or ``{}'' immediately following the command.

% The sections included here are required. Additional sections and subsections may be added as necessary.
\section{Introduction}
\label{sec:introduction}
  % The introduction is a place to give background and context before diving into the subject matter.
  % Establish context for the work you are about to propose and the main ideas of the proposition itself.

\IEEEPARstart{R}{IT} SPEX has direct ties to the RIT Observatory though one of the primary advisors and a mutual interest in astronomy.
Despite this, both groups have had little collaboration.
The RIT SPEX Astrodynamics team has some equipment that is measurable to the equipment in the observatory, but has trouble with operation and lack of experience.
The RIT Observatory has a need for students to work on engineering projects, which RIT SPEX Astrodynamics can provide.
RIT SPEX has always had members interested in observing using various telescopes, but these members have little training.
There have already been projects that gave the team insight into the equipment behind observing.
These include: building and modifying a rolling mount for its 12'' Meade telescope, building mounts for an Orion sighting scope, building a barn door camera mount, and general maintenance on all telescopes.
The team is looking for more projects similarly to expand their skill sets.

\section{Primary Objective}
\label{sec:primary-obj}
  % At the end of the day, whether the project ``succeeds'' or ``fails'' is judged against the objectives it sought to meet.
  % Note that results that contradict expectations/hypotheses are not failures if the scientific \& engineering methods are followed along the way.
  % Sometimes our expectations are wrong and that can be just as successful as getting data we thought we'd see.
  % What matters are what questions you intend to answer.
  % This is the main purpose or main goal the project hopes to achieve.

  There are three projects selected to be worked on in the fall semester.
  These projects are the first steps towards improving the observatory's capabilities and a convenient fully functioning automated dome system.

  \subsection{Roll Off Motor Repair}
\label{subsec:rollOffMotorRepair}
The observatory has two domes, yet one is not very widely used because of the system to remove the roof.
The current system is a hand crank which rolls the entire roof off of the structure.
This is inconvenient and must be done with two people.
Years ago, a motor system was installed so that the crank system would not be needed.
This motor system was deemed too fast by RIT Safety and forbidden to be used.
The faculty at the observatory would like to use this dome more frequently and believe this would be possible if the motor is repaired, slowed down, and a warning system installed.
Work is currently being done to assess the state of the motor and SPEX students would take over where this is left off.

\subsection{Latch Arm Repair}
\label{subsec:latchArmRepair}
In an attempt to start the automation of the primary dome at the observatory, a MSD team began work on a system to be integrated into the current control system.
Part of this system was a latch arm assembly which would pull on the chain that separates the upper and lower parts of the dome slit cover.
Currently, a faculty member has to climb a ladder, lean over high voltage power lines, and unlatch the chain.
This latch arm assembly would be mounted on the dome, and allow a user to remotely unlatch the chain when needed for observing.
The assembly is all built, but there is a problem with power usage.
The faculty at the observatory would like this assembly fixed and installed.

\subsection{12'' Meade Modifications}
\label{subsec:meadeModifications}
The faculty have requested small modifications to the 12'' Meade telescope that would greatly improve the convenience and quality of observing.
One major modification is installation of LEDs onto the tube of the telescope to be used to take dome flat calibration images, which is currently done with an LED mounted behind the telescope on the dome wall.
Adding another set screw to hold the heavy CCD camera that is mounted into the telescope has also been requested.
Modifying the safety system for holding the CCD camera in case it falls is also a task that could be worked on.

\section{Benefit to SPEX}
\label{sec:benefit}
% One of the core values of SPEX is to provide opportunities for academic and professional growth for its members,
% and to challenge them with interesting projects.
% In this section, explain how the project would benefit SPEX members as students,
% space enthusiasts, and young professionals.

By working with the equipment and the facility, SPEX students will have access to the observatory and gain experience with the equipment.
Since SPEX is an organization for all students of all majors, students with an interest in astronomy and telescopes who may not have the opportunity to take astronomy classes will have access to the observatory.
Ideally, this collaboration benefits SPEX, the observatory, and all students at RIT.

% Below I have used subsections to identify key ideas in this section. These particular subsections are not required as part of the SPEX Standard, but serve as an example of using subsections in a text.

\subsection{Engineering Skills}
\label{subsec:engineeringSkills}
Firstly, these projects will help students develop key engineering skills, as do most projects that SPEX takes on.
These projects will take students through the engineering process from beginning to end, involving problem solving, slight planning, research, and implementation.
Students will have the opportunity to be involved in every step of the way if desired.
No high level engineering skills are required to complete these projects, but students may learn or enhance their experience with Arduino, circuits, hand tools, power tools, integration, and control systems.

\subsection{Observational Skills}
\label{subsec:observationalSkills}
Secondly, while working with the equipment at the observatory, students will become more familiar with the workings of the telescopes.
This will help students understand what happens behind taking observations, and directly affect their observational skills with the SPEX telescopes.
They will become more comfortable with the equipment itself.

\subsection{Accessibility}
\label{subsec:plug-n-play}
SPEX is an organization for all students of all majors.
Students with an interest in astronomy and telescopes who may not have the opportunity to take astronomy classes will have access to the observatory.
If steps towards remote observing are taken and eventually implemented, it will lighten the burden of observing on the faculty, which in turn opens the observatory to more students.
A strong bond with the observatory also opens up more opportunities there for SPEX students specifically.

\section{Implementation}
\label{sec:implementation}
  % What path do you anticipate the project to take?

 The projects can be worked on simultaneously, or one after the other depending on the number of interested members.
 Ideally, projects that require more work in the outdoors would want to be completed first to avoid weather concerns.
 If these main first projects are implemented, there are more projects that can be started in a new PDD, or these can be enhanced and continued in the spring.
 Being the first semester of these ongoing projects, the goal is to complete these smaller tasks and work out the process of collaborating with the faculty at the observatory.

\subsection{Deliverables}
\label{subsec:deliverables}
  % When all is said and done, what will you have to show for it?
  % Examples: Hardware, software, poster, ImagineRIT demo, presentations, technical papers...
  Physical deliverables are the outcome of these projects.
  By the end of the semester the roll off motor should be repaired, slowed down, and an warning system installed.
  The motor latch arm should be installed correctly and its power usage limited.
  An LED for dome flat calibration images should be installed onto the 12'' Meade as well as any other small modifications.

\subsection{Milestones}
\label{subsec:milestones}
  % Be as detailed as you can, but it's okay if there are unknowns.
  % At the very least, specify how many semester you expect the project to take until it reaches completion.
  The projects designated for this semester have been selected because it is expected that the tasks can be feasibly completed within one semester and without as much experience with the facility.
  Two potential timelines are outlined in \autoref{tab:timeline1} and \autoref{tab:timeline2}.
  The timelines depend on the number of people available to work on the projects.

\begin{table}[hb!]
  % this table is too wide for the two-column format, so we let it expand across both columns
  % we haven't told LaTeX where to put this so it'll find the best place.
      \caption{Notional Timeline of Project Milestones I}
      \centering
      \begin{tabularx}{\linewidth}{lXl}
          % READ THIS!! https://www.inf.ethz.ch/personal/markusp/teaching/guides/guide-tables.pdf
          \toprule % line on top external edge of table
          % Separate cells in a row with &, move to the next row with \\
          Phase & Task & Duration \\
          \midrule % line separating two internal rows
          0 & Gather documentation on all equipment and preliminary work & Summer \\
          1 & Review existing designs and materials for roll off motor, telescope modification,and latch arm & 2 weeks or less \\
          2 & Systems development \\
          & Repair existing roll off motor \\
          & Design Schematic to fix latch arm \\
          & Design mounting for selected LED \\
          & Determine placement of modifications \\
          & Solder modifications to latch arm controls \\
          & Design roll off warning systems \\
          & Order parts for schematics & 5 weeks \\
          3 & Mount systems at observatory & 2 weeks \\
          4 & Systems testing & 1 week \\
          5 & Generate documentation and delivery to SPEX & 1 week \\
          % LaTeX doesn't really like multi-line cell contents. Try to keep the text in each cell concise!
          \bottomrule
      \end{tabularx}
  \label{tab:timeline1}
  \end{table}

  \begin{table}[hb!]
    % this table is too wide for the two-column format, so we let it expand across both columns
    % we haven't told LaTeX where to put this so it'll find the best place.
        \caption{Notional Timeline of Project Milestones II}
        \centering
        \begin{tabularx}{\linewidth}{lXl}
            % READ THIS!! https://www.inf.ethz.ch/personal/markusp/teaching/guides/guide-tables.pdf
            \toprule % line on top external edge of table
            % Separate cells in a row with &, move to the next row with \\
            Phase & Task & Duration \\
            \midrule % line separating two internal rows
            0 & Gather documentation on all equipment and preliminary work & Summer \\
            1 & Review existing designs and materials for roll off motor & 1 week \\
            2 & System development of roll off motor
            & Repair existing roll off motor  \\
            & Design roll off warning systems \\
            & Order parts for schematics \\
            & Install system at observatory & 4 weeks \\
            3 & Review existing designs and materials for latch arm & 1 week \\
            4 & System development of latch arm assembly \\
            & Get latch arm working in current state \\
            & Design schematic to fix power usage \\
            & Order parts for schematics \\
            & Solder modifications to latch arm controls \\
            & Install system at observatory & 3 weeks \\
            5 & Review existing designs and materials for telescope modifications & 1 week \\
            6 & System development of latch arm assembly \\
            & Design mounting for selected LED \\
            & Determine placement of modifications \\
            & Install system at observatory & 2 weeks \\
            7 & Systems testing & 2 weeks \\
            8 & Generate documentation and delivery to SPEX & 1 week \\
            % LaTeX doesn't really like multi-line cell contents. Try to keep the text in each cell concise!
            \bottomrule
        \end{tabularx}
    \label{tab:timeline2}
    \end{table}

\section{Externalities}
  % Things not directly related to the work or outcomes, but related to the project as a whole.
\subsection{Prerequisite Skills}
  % Which skills do team members need to have before work can start (not including skills that will be learned ``on the job'')?
  Leadership with the prerequisite skills has been determined.
  It would be beneficial for some students to already have basic engineering and hand tool skills.
  Knowledge of Arduino code and basic electrical circuits is also beneficial.
  These skills do not need to be present in all group members as these projects are an opportunity for new members to learn these skills.
  For the roll off motor repair, someone with some motor knowledge would be crucial.

\subsection{Funding Requirements}
  % Estimate costs that would be needed to meet objectives.
  Primary funding for these projects will come from the Observatory itself.
  If extra funds are needed, RIT SPEX Astrodynamics may apply for SG funding due to their newly developed SG club status.
  \autoref{tab:costEstimate} outlines the estimated costs of the projects.
  The costs have been overestimated for planning purposes.

  \begin{table}[hb!]
    % this table is too wide for the two-column format, so we let it expand across both columns
    % we haven't told LaTeX where to put this so it'll find the best place.
        \caption{Estimated Budget}
        \centering
        \begin{tabular}{ll}
          % READ THIS!! https://www.inf.ethz.ch/personal/markusp/teaching/guides/guide-tables.pdf
          \toprule % line on top external edge of table
          % Separate cells in a row with &, move to the next row with \\
          Project & Cost \\
          \midrule % line separating two internal rows
          Roll Off Motor Repair & \$500 \\
          Motor Latch Arm & \$100 \\
          12" Telescope Modifications & \$100 \\
          % LaTeX doesn't really like multi-line cell contents. Try to keep the text in each cell concise!
          \bottomrule
        \end{tabular}
    \label{tab:costEstimate}
    \end{table}

\subsection{Faculty Support}
  % Identify faculty that will be involved (or would need to be involved) to meet objectives.
  % Note that if a professor is the Principal Investigator (P.I.) for a project, there still needs to be a student as the SPEX Project Champion.
  Direct support from Dr. Richmond and Dr. Connelly is expected.
  Access to the facility is highly dependent on one of the above being present.
  Both faculty are very familiar with the equipment, telescopes, and facility.
  They also have experience with what is typical for systems at other observatories, which may be beneficial when working on these projects.
  Dr. Richmond has been modifying equipment at the observatory for decades.
  Dr. Connelly understands the structure and capabilities of RIT SPEX through being an advisor.
  Both are great resources for guidance during the completion of the projects.
  As all professors, they are juggling multiple responsibilities at one time, which can complicate scheduling time for this project.
  Even so, they understand the needs of the observatory and can also be seen as the primary customers for these projects, so it is important to keep them informed through the engineering process

\subsection{Long-Term Vision}
\label{sec:vision}
The long term goal of these projects is to work towards a fully automated dome system.
This would enable remote observing for the Observatory and enable increased access to more students.
Students will get familiar with the technology behind observing and enhance their basic engineering skills.
As projects get completed, a strong bond between the RIT Observatory and RIT SPEX can be established.

\section*{Acknowledgements}
The author would like to thank the advisors of SPEX, Evan Putnam and Marilyn Wolbert for continuing the Astrodynamics team for the 2017--2018 school year, the faculty at the observatory for reaching out about these projects, and the SPEX admins for their extra effort put in to make this project process possible.
\end{document}
