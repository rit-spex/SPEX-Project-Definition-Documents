%% *************************************************************************
%%
%% This is an RIT Space Exploration Standard defining guidelines for content
%% and formatting of project proposals.
%%
%% The document template SPEXformat.cls is based on the official IEEE LaTeX
%% class for authors of the Institute of Electrical and Electronics Engineers
%% (IEEE) Transactions journals and conferences.
%%
%% *************************************************************************
\documentclass[journal]{SPEXformat}
\usepackage{blindtext}
\usepackage{graphicx}
\usepackage{nomencl}
\makenomenclature{}
\title{RIT Space Exploration Project Proposal Standard Format and Sample Content}
\author{
  % List the authors of the proposal. The Champion should go first.
  % The `$ $` markers tell \LaTeX to treat the text inside to be treated as a math expression. This way you can use operators like `^` to place characters as superscripts.
  % The `\thanks` command puts the contents inside those brackets in a footnote at the bottom of the first page. Technically speaking, `\thanks` is just a specially formatted footnote. Read here for a more in-depth explanation: http://tex.stackexchange.com/questions/61303/using-author-and-thanks-for-authors-with-common-affiliations
  Philip~Linden$^{\dagger}$\thanks{$^{\dagger}$BS/MEng '17, Mechanical Engineering},
  Austin~Bodzas$^{*}$\thanks{$^{*}$BS '19, Computer Science}
}
  % page header for pages other than cover page
  \markboth{Project Proposal Standard}%
  {Linden \MakeLowercase{\textit{et al.}}: RIT Space Exploration}

\begin{document}
\maketitle
% correct bad hyphenation here, separated by spaces
\hyphenation{explor-ation}

\begin{abstract}
% The abstract is a brief summary of the proposal. Typically it includes the purpose of the proposal, key goals or objectives, and justifications.
% Be sure not to confuse the abstract with the introduction.
% It is easiest to write the abstract after the rest of the paper has been written.
% That way you can choose key information from the sections that you've already completed and string them together in the abstract.
% Consider the abstract to be your elevator pitch to anyone reading this proposal.
% What are they reading?
% What is the goal?
% Why is it worth my time?
% The abstract is what will show up in Google results and other search engines, and what people will read when they are deciding what is worth their time and brain power.
A standard format for SPEX Project Proposals is key to organize and document the many projects members of RIT Space Exploration wish to pursue. The goal of the SPEX Standard is to organize, refine, and archive space exploration research.
Documentation is vital to sharing and maintaining the wealth of ideas and information developed by all students at RIT.\@
SPEX Project Proposals aim to provide a foundation for new projects to grow, or premature projects to develop months or years in the future.
A standard for project proposals and reports shall provide SPEX with a robust method to maintain a healthy ecosystem of projects in all stages of development including the event where a SPEX member goes on co-op or graduates.
\end{abstract}

\label{sec:nomenclature}
% If you include mathematical expressions or express variables in the proposal, list them with their corresponding definitions here as a list.
% Read more about advanced nomenclature formatting here: https://www.sharelatex.com/learn/Nomenclatures

\nomenclature{RIT}{Rochester Institute of Technology}
\nomenclature{SPEX}{RIT Space Exploration}
\nomenclature{SPP}{SPEX Project Proposal}
\nomenclature{$h$}{Planck constant}
\nomenclature{$c$}{Speed of light in a vacuum inertial frame}
% Note that math terms and non-math terms are separated and alphabetized, regardless of the order in which they are defined. (Recall terms $like this$ are in the math environment)
\printnomenclature{}

\section{Introduction}
\label{sec:introduction}
% The introduction is a place to give background and context before diving into the subject matter.
Examples of proper formatting, organizational techniques and content make writing SPEX Project Proposals as easy and painless as possible.
Writing documentation such as proposals and reports is a lot of work, but it supports the continued growth of knowledge and experience in science and engineering for SPEX as a whole.

\end{document}
